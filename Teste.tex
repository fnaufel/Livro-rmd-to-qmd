% Options for packages loaded elsewhere
\PassOptionsToPackage{unicode}{hyperref}
\PassOptionsToPackage{hyphens}{url}
\PassOptionsToPackage{dvipsnames,svgnames,x11names}{xcolor}
%
\documentclass[
  letterpaper,
  DIV=11,
  numbers=noendperiod]{scrreprt}

\usepackage{amsmath,amssymb}
\usepackage{iftex}
\ifPDFTeX
  \usepackage[T1]{fontenc}
  \usepackage[utf8]{inputenc}
  \usepackage{textcomp} % provide euro and other symbols
\else % if luatex or xetex
  \usepackage{unicode-math}
  \defaultfontfeatures{Scale=MatchLowercase}
  \defaultfontfeatures[\rmfamily]{Ligatures=TeX,Scale=1}
\fi
\usepackage{lmodern}
\ifPDFTeX\else  
    % xetex/luatex font selection
\fi
% Use upquote if available, for straight quotes in verbatim environments
\IfFileExists{upquote.sty}{\usepackage{upquote}}{}
\IfFileExists{microtype.sty}{% use microtype if available
  \usepackage[]{microtype}
  \UseMicrotypeSet[protrusion]{basicmath} % disable protrusion for tt fonts
}{}
\makeatletter
\@ifundefined{KOMAClassName}{% if non-KOMA class
  \IfFileExists{parskip.sty}{%
    \usepackage{parskip}
  }{% else
    \setlength{\parindent}{0pt}
    \setlength{\parskip}{6pt plus 2pt minus 1pt}}
}{% if KOMA class
  \KOMAoptions{parskip=half}}
\makeatother
\usepackage{xcolor}
\setlength{\emergencystretch}{3em} % prevent overfull lines
\setcounter{secnumdepth}{3}
% Make \paragraph and \subparagraph free-standing
\ifx\paragraph\undefined\else
  \let\oldparagraph\paragraph
  \renewcommand{\paragraph}[1]{\oldparagraph{#1}\mbox{}}
\fi
\ifx\subparagraph\undefined\else
  \let\oldsubparagraph\subparagraph
  \renewcommand{\subparagraph}[1]{\oldsubparagraph{#1}\mbox{}}
\fi

\usepackage{color}
\usepackage{fancyvrb}
\newcommand{\VerbBar}{|}
\newcommand{\VERB}{\Verb[commandchars=\\\{\}]}
\DefineVerbatimEnvironment{Highlighting}{Verbatim}{commandchars=\\\{\}}
% Add ',fontsize=\small' for more characters per line
\usepackage{framed}
\definecolor{shadecolor}{RGB}{241,243,245}
\newenvironment{Shaded}{\begin{snugshade}}{\end{snugshade}}
\newcommand{\AlertTok}[1]{\textcolor[rgb]{0.68,0.00,0.00}{#1}}
\newcommand{\AnnotationTok}[1]{\textcolor[rgb]{0.37,0.37,0.37}{#1}}
\newcommand{\AttributeTok}[1]{\textcolor[rgb]{0.40,0.45,0.13}{#1}}
\newcommand{\BaseNTok}[1]{\textcolor[rgb]{0.68,0.00,0.00}{#1}}
\newcommand{\BuiltInTok}[1]{\textcolor[rgb]{0.00,0.23,0.31}{#1}}
\newcommand{\CharTok}[1]{\textcolor[rgb]{0.13,0.47,0.30}{#1}}
\newcommand{\CommentTok}[1]{\textcolor[rgb]{0.37,0.37,0.37}{#1}}
\newcommand{\CommentVarTok}[1]{\textcolor[rgb]{0.37,0.37,0.37}{\textit{#1}}}
\newcommand{\ConstantTok}[1]{\textcolor[rgb]{0.56,0.35,0.01}{#1}}
\newcommand{\ControlFlowTok}[1]{\textcolor[rgb]{0.00,0.23,0.31}{#1}}
\newcommand{\DataTypeTok}[1]{\textcolor[rgb]{0.68,0.00,0.00}{#1}}
\newcommand{\DecValTok}[1]{\textcolor[rgb]{0.68,0.00,0.00}{#1}}
\newcommand{\DocumentationTok}[1]{\textcolor[rgb]{0.37,0.37,0.37}{\textit{#1}}}
\newcommand{\ErrorTok}[1]{\textcolor[rgb]{0.68,0.00,0.00}{#1}}
\newcommand{\ExtensionTok}[1]{\textcolor[rgb]{0.00,0.23,0.31}{#1}}
\newcommand{\FloatTok}[1]{\textcolor[rgb]{0.68,0.00,0.00}{#1}}
\newcommand{\FunctionTok}[1]{\textcolor[rgb]{0.28,0.35,0.67}{#1}}
\newcommand{\ImportTok}[1]{\textcolor[rgb]{0.00,0.46,0.62}{#1}}
\newcommand{\InformationTok}[1]{\textcolor[rgb]{0.37,0.37,0.37}{#1}}
\newcommand{\KeywordTok}[1]{\textcolor[rgb]{0.00,0.23,0.31}{#1}}
\newcommand{\NormalTok}[1]{\textcolor[rgb]{0.00,0.23,0.31}{#1}}
\newcommand{\OperatorTok}[1]{\textcolor[rgb]{0.37,0.37,0.37}{#1}}
\newcommand{\OtherTok}[1]{\textcolor[rgb]{0.00,0.23,0.31}{#1}}
\newcommand{\PreprocessorTok}[1]{\textcolor[rgb]{0.68,0.00,0.00}{#1}}
\newcommand{\RegionMarkerTok}[1]{\textcolor[rgb]{0.00,0.23,0.31}{#1}}
\newcommand{\SpecialCharTok}[1]{\textcolor[rgb]{0.37,0.37,0.37}{#1}}
\newcommand{\SpecialStringTok}[1]{\textcolor[rgb]{0.13,0.47,0.30}{#1}}
\newcommand{\StringTok}[1]{\textcolor[rgb]{0.13,0.47,0.30}{#1}}
\newcommand{\VariableTok}[1]{\textcolor[rgb]{0.07,0.07,0.07}{#1}}
\newcommand{\VerbatimStringTok}[1]{\textcolor[rgb]{0.13,0.47,0.30}{#1}}
\newcommand{\WarningTok}[1]{\textcolor[rgb]{0.37,0.37,0.37}{\textit{#1}}}

\providecommand{\tightlist}{%
  \setlength{\itemsep}{0pt}\setlength{\parskip}{0pt}}\usepackage{longtable,booktabs,array}
\usepackage{calc} % for calculating minipage widths
% Correct order of tables after \paragraph or \subparagraph
\usepackage{etoolbox}
\makeatletter
\patchcmd\longtable{\par}{\if@noskipsec\mbox{}\fi\par}{}{}
\makeatother
% Allow footnotes in longtable head/foot
\IfFileExists{footnotehyper.sty}{\usepackage{footnotehyper}}{\usepackage{footnote}}
\makesavenoteenv{longtable}
\usepackage{graphicx}
\makeatletter
\def\maxwidth{\ifdim\Gin@nat@width>\linewidth\linewidth\else\Gin@nat@width\fi}
\def\maxheight{\ifdim\Gin@nat@height>\textheight\textheight\else\Gin@nat@height\fi}
\makeatother
% Scale images if necessary, so that they will not overflow the page
% margins by default, and it is still possible to overwrite the defaults
% using explicit options in \includegraphics[width, height, ...]{}
\setkeys{Gin}{width=\maxwidth,height=\maxheight,keepaspectratio}
% Set default figure placement to htbp
\makeatletter
\def\fps@figure{htbp}
\makeatother

% Soul package to handle highlighting (see hl.py3 filter)
\usepackage{soul}

\KOMAoption{captions}{tableheading}
\makeatletter
\@ifpackageloaded{tcolorbox}{}{\usepackage[skins,breakable]{tcolorbox}}
\@ifpackageloaded{fontawesome5}{}{\usepackage{fontawesome5}}
\definecolor{quarto-callout-color}{HTML}{909090}
\definecolor{quarto-callout-note-color}{HTML}{0758E5}
\definecolor{quarto-callout-important-color}{HTML}{CC1914}
\definecolor{quarto-callout-warning-color}{HTML}{EB9113}
\definecolor{quarto-callout-tip-color}{HTML}{00A047}
\definecolor{quarto-callout-caution-color}{HTML}{FC5300}
\definecolor{quarto-callout-color-frame}{HTML}{acacac}
\definecolor{quarto-callout-note-color-frame}{HTML}{4582ec}
\definecolor{quarto-callout-important-color-frame}{HTML}{d9534f}
\definecolor{quarto-callout-warning-color-frame}{HTML}{f0ad4e}
\definecolor{quarto-callout-tip-color-frame}{HTML}{02b875}
\definecolor{quarto-callout-caution-color-frame}{HTML}{fd7e14}
\makeatother
\makeatletter
\@ifpackageloaded{bookmark}{}{\usepackage{bookmark}}
\makeatother
\makeatletter
\@ifpackageloaded{caption}{}{\usepackage{caption}}
\AtBeginDocument{%
\ifdefined\contentsname
  \renewcommand*\contentsname{Índice}
\else
  \newcommand\contentsname{Índice}
\fi
\ifdefined\listfigurename
  \renewcommand*\listfigurename{Lista de Figuras}
\else
  \newcommand\listfigurename{Lista de Figuras}
\fi
\ifdefined\listtablename
  \renewcommand*\listtablename{Lista de Tabelas}
\else
  \newcommand\listtablename{Lista de Tabelas}
\fi
\ifdefined\figurename
  \renewcommand*\figurename{Figura}
\else
  \newcommand\figurename{Figura}
\fi
\ifdefined\tablename
  \renewcommand*\tablename{Tabela}
\else
  \newcommand\tablename{Tabela}
\fi
}
\@ifpackageloaded{float}{}{\usepackage{float}}
\floatstyle{ruled}
\@ifundefined{c@chapter}{\newfloat{codelisting}{h}{lop}}{\newfloat{codelisting}{h}{lop}[chapter]}
\floatname{codelisting}{Listagem}
\newcommand*\listoflistings{\listof{codelisting}{Lista de Listagens}}
\makeatother
\makeatletter
\makeatother
\makeatletter
\@ifpackageloaded{caption}{}{\usepackage{caption}}
\@ifpackageloaded{subcaption}{}{\usepackage{subcaption}}
\makeatother
\ifLuaTeX
\usepackage[bidi=basic]{babel}
\else
\usepackage[bidi=default]{babel}
\fi
\babelprovide[main,import]{portuguese}
% get rid of language-specific shorthands (see #6817):
\let\LanguageShortHands\languageshorthands
\def\languageshorthands#1{}
\ifLuaTeX
  \usepackage{selnolig}  % disable illegal ligatures
\fi
\usepackage{bookmark}

\IfFileExists{xurl.sty}{\usepackage{xurl}}{} % add URL line breaks if available
\urlstyle{same} % disable monospaced font for URLs
\hypersetup{
  pdftitle={Teste},
  pdfauthor={Fernando Náufel},
  pdflang={pt},
  colorlinks=true,
  linkcolor={blue},
  filecolor={Maroon},
  citecolor={Blue},
  urlcolor={Blue},
  pdfcreator={LaTeX via pandoc}}

\title{Teste}
\author{Fernando Náufel}
\date{25/01/2024 18:40}

\begin{document}
\maketitle

% Bold title in callout boxes
\tcbset{fonttitle=\bfseries}

\renewcommand*\contentsname{Índice}
{
\hypersetup{linkcolor=}
\setcounter{tocdepth}{2}
\tableofcontents
}
\bookmarksetup{startatroot}

\chapter*{Apresentação}\label{apresentacao}
\addcontentsline{toc}{chapter}{Apresentação}

\markboth{Apresentação}{Apresentação}

\begin{tcolorbox}[enhanced jigsaw, coltitle=black, colbacktitle=quarto-callout-caution-color!10!white, title=\textcolor{quarto-callout-caution-color}{\faFire}\hspace{0.5em}{Cuidado}, toprule=.15mm, leftrule=.75mm, opacityback=0, colback=white, arc=.35mm, breakable, bottomtitle=1mm, left=2mm, toptitle=1mm, titlerule=0mm, rightrule=.15mm, bottomrule=.15mm, opacitybacktitle=0.6, colframe=quarto-callout-caution-color-frame]

Este material ainda está em construção.

Pode haver mudanças a qualquer momento.

Verifique, no rodapé da página \emph{web} ou na capa do arquivo pdf, a
data desta versão.

\end{tcolorbox}

\includegraphics{images/640px-Nightingale-mortality.jpg}

Este livro/\emph{site} foi iniciado em 2020, durante a pandemia de
COVID-19, quando a Universidade Federal Fluminense (UFF) funcionou em
regime de ensino remoto durante mais de um ano.

Para atender os alunos do curso de Probabilidade e Estatística do curso
de graduação em Ciência da Computação da UFF, decidi gravar aulas em
vídeo e disponibilizar os arquivos usados nelas. Foram esses arquivos
que deram origem a este livro/\emph{site}.

Este livro/\emph{site} foi construído para pessoas que já saibam
programar, embora não necessariamente em R.

Para tirar o máximo proveito deste material, você deve fazer o seguinte:

\begin{enumerate}
\def\labelenumi{\arabic{enumi}.}
\item
  Assistir aos vídeos contidos em cada capítulo. A \emph{playlist}
  completa está em
  https://www.youtube.com/playlist?list=PL7SRLwLs7ocaV-Y1vrVU3W7mZnnS0qkWV.
\item
  Instalar o R no seu computador ou abrir uma conta no RStudio Cloud,
  para poder usar o R \emph{online}. Você encontra instruções para fazer
  isto no \hyperref[rintro]{capítulo de introdução a R}.
\item
  Baixar, \href{https://github.com/fnaufel/probestr}{neste repositório
  do Github}, o código-fonte deste livro/\emph{site}, para poder rodar e
  alterar os exemplos.
\item
  Seguir os \emph{links} para outras fontes \emph{online} que abordam
  assuntos que não são cobertos em detalhes neste curso.
\item
  Fazer os exercícios. Ao longo do tempo, acrescentarei \emph{links}
  para vídeos explicando as soluções.
\end{enumerate}

\begin{tcolorbox}[enhanced jigsaw, coltitle=black, colbacktitle=quarto-callout-important-color!10!white, title=\textcolor{quarto-callout-important-color}{\faExclamation}\hspace{0.5em}{Importante}, toprule=.15mm, leftrule=.75mm, opacityback=0, colback=white, arc=.35mm, breakable, bottomtitle=1mm, left=2mm, toptitle=1mm, titlerule=0mm, rightrule=.15mm, bottomrule=.15mm, opacitybacktitle=0.6, colframe=quarto-callout-important-color-frame]

{\hl{Se você estiver lendo este material na \emph{web}, você pode clicar
nos comandos e funções que aparecem nos blocos de código em R}} para
abrir páginas da documentação sobre eles.

Se você preferir ler este livro em pdf, ou se quiser imprimi-lo,
\href{https://github.com/fnaufel/probestr/blob/master/docs/probestr.pdf}{faça
o \emph{download} do arquivo aqui}.

\end{tcolorbox}

\section*{Referências recomendadas}\label{refrec}
\addcontentsline{toc}{section}{Referências recomendadas}

\markright{Referências recomendadas}

\subsection*{Em português}\label{em-portuguuxeas}
\addcontentsline{toc}{subsection}{Em português}

\begin{itemize}
\item
  Sillas Gonzaga, \emph{Introdução a R para Visualização e Apresentação
  de Dados},
  http://sillasgonzaga.com/material/curso\_visualizacao/index.html
\item
  Allan Vieira de Castro Quadros, \emph{Introdução à Análise de Dados em
  R utilizando Tidyverse}, https://allanvc.github.io/book\_IADR-T/
\item
  Paulo Felipe de Oliveira, Saulo Guerra, Robert McDonnel, \emph{Ciência
  de Dados com R -- Introdução}, https://cdr.ibpad.com.br/index.html
\item
  Curso R, \emph{Ciência de Dados em R}, https://livro.curso-r.com/
\end{itemize}

\subsection*{Em inglês}\label{em-ingluxeas}
\addcontentsline{toc}{subsection}{Em inglês}

\begin{itemize}
\item
  Garrett Grolemund, Hadley Wickham, \emph{R for Data Science},
  https://r4ds.had.co.nz/
\item
  Chester Ismay, Albert Y. Kim, \emph{A ModernDive into R and the
  Tidyverse}, https://moderndive.com/
\end{itemize}

\section*{Exercício}\label{exercuxedcio}
\addcontentsline{toc}{section}{Exercício}

\markright{Exercício}

\begin{enumerate}
\def\labelenumi{\arabic{enumi}.}
\tightlist
\item
  Pesquise sobre a imagem do início deste capítulo. Ela foi criada em
  1858 por Florence Nightingale.
\end{enumerate}

\bookmarksetup{startatroot}

\chapter{\texorpdfstring{Introdução ao
\texttt{tidyverse}}{Introdução ao tidyverse}}\label{introduuxe7uxe3o-ao-tidyverse}

\begin{tcolorbox}[enhanced jigsaw, coltitle=black, colbacktitle=quarto-callout-tip-color!10!white, title=\textcolor{quarto-callout-tip-color}{\faLightbulb}\hspace{0.5em}{Dica}, toprule=.15mm, leftrule=.75mm, opacityback=0, colback=white, arc=.35mm, breakable, bottomtitle=1mm, left=2mm, toptitle=1mm, titlerule=0mm, rightrule=.15mm, bottomrule=.15mm, opacitybacktitle=0.6, colframe=quarto-callout-tip-color-frame]

Busque mais informações sobre os pacotes que compõem o
\texttt{tidyverse} \hyperref[refrec]{nas referências recomendadas}.

\end{tcolorbox}

\section{\texorpdfstring{Criando uma
\emph{tibble}}{Criando uma tibble}}\label{criando-uma-tibble}

\begin{itemize}
\item
  Uma \emph{tibble} é uma tabela retangular.
\item
  {\hl{Cada coluna é um vetor}}:

\begin{Shaded}
\begin{Highlighting}[]
\NormalTok{cores }\OtherTok{\textless{}{-}} \FunctionTok{tibble}\NormalTok{(}
  \AttributeTok{pessoa =} \FunctionTok{c}\NormalTok{(}\StringTok{\textquotesingle{}João\textquotesingle{}}\NormalTok{, }\StringTok{\textquotesingle{}Maria\textquotesingle{}}\NormalTok{, }\StringTok{\textquotesingle{}Pedro\textquotesingle{}}\NormalTok{, }\StringTok{\textquotesingle{}Ana\textquotesingle{}}\NormalTok{),}
  \StringTok{\textquotesingle{}cor favorita\textquotesingle{}} \OtherTok{=} \FunctionTok{c}\NormalTok{(}\StringTok{\textquotesingle{}azul\textquotesingle{}}\NormalTok{, }\StringTok{\textquotesingle{}rosa\textquotesingle{}}\NormalTok{, }\StringTok{\textquotesingle{}preto\textquotesingle{}}\NormalTok{, }\StringTok{\textquotesingle{}branco\textquotesingle{}}\NormalTok{)}
\NormalTok{)}
\end{Highlighting}
\end{Shaded}

\begin{Shaded}
\begin{Highlighting}[]
\NormalTok{cores}
\end{Highlighting}
\end{Shaded}

\begin{verbatim}
# A tibble: 4 x 2
  pessoa `cor favorita`
  <chr>  <chr>         
1 João   azul          
2 Maria  rosa          
3 Pedro  preto         
4 Ana    branco        
\end{verbatim}
\item
  Isto é um pouco diferente da maneira como estamos acostumados a ver
  tabelas (como uma coleção de linhas, em vez de uma coleção de
  colunas).
\item
  A função \texttt{tribble} permite a entrada de forma mais natural,
  linha a linha. {\hl{Lembre-se de usar
  {\mbox{\texttt{\textasciitilde{}}}} antes dos nomes das colunas.}}

\begin{Shaded}
\begin{Highlighting}[]
\NormalTok{cores }\OtherTok{\textless{}{-}} \FunctionTok{tribble}\NormalTok{(}
  \SpecialCharTok{\textasciitilde{}}\NormalTok{pessoa, }\SpecialCharTok{\textasciitilde{}}\StringTok{\textquotesingle{}cor favorita\textquotesingle{}}\NormalTok{,}
   \StringTok{"João"}\NormalTok{,          }\StringTok{"azul"}\NormalTok{,}
  \StringTok{"Maria"}\NormalTok{,          }\StringTok{"rosa"}\NormalTok{,}
  \StringTok{"Pedro"}\NormalTok{,         }\StringTok{"preto"}\NormalTok{,}
    \StringTok{"Ana"}\NormalTok{,        }\StringTok{"branco"}
\NormalTok{)}
\end{Highlighting}
\end{Shaded}

\begin{Shaded}
\begin{Highlighting}[]
\NormalTok{cores}
\end{Highlighting}
\end{Shaded}

\begin{verbatim}
# A tibble: 4 x 2
  pessoa `cor favorita`
  <chr>  <chr>         
1 João   azul          
2 Maria  rosa          
3 Pedro  preto         
4 Ana    branco        
\end{verbatim}

  \begin{tcolorbox}[enhanced jigsaw, coltitle=black, colbacktitle=quarto-callout-caution-color!10!white, title=\textcolor{quarto-callout-caution-color}{\faFire}\hspace{0.5em}{Cuidado}, toprule=.15mm, leftrule=.75mm, opacityback=0, colback=white, arc=.35mm, breakable, bottomtitle=1mm, left=2mm, toptitle=1mm, titlerule=0mm, rightrule=.15mm, bottomrule=.15mm, opacitybacktitle=0.6, colframe=quarto-callout-caution-color-frame]

  Mesmo que você crie uma \emph{tibble} linha a linha, o R vai continuar
  tratando sua \emph{tibble} como uma coleção de colunas.

  É importante lembrar disto para entender a forma como R manipula estas
  tabelas.

  \end{tcolorbox}
\item
  Se uma coluna não puder ser armazenada em um vetor, a coluna será uma
  lista (com vetores como elementos):

\begin{Shaded}
\begin{Highlighting}[]
\NormalTok{cores }\OtherTok{\textless{}{-}} \FunctionTok{tibble}\NormalTok{(}
  \AttributeTok{pessoa =} \FunctionTok{c}\NormalTok{(}\StringTok{\textquotesingle{}João\textquotesingle{}}\NormalTok{, }\StringTok{\textquotesingle{}Maria\textquotesingle{}}\NormalTok{, }\StringTok{\textquotesingle{}Pedro\textquotesingle{}}\NormalTok{, }\StringTok{\textquotesingle{}Ana\textquotesingle{}}\NormalTok{),}
  \StringTok{\textquotesingle{}cor favorita\textquotesingle{}} \OtherTok{=} \FunctionTok{list}\NormalTok{(}
    \FunctionTok{c}\NormalTok{(}\StringTok{\textquotesingle{}azul\textquotesingle{}}\NormalTok{, }\StringTok{\textquotesingle{}roxo\textquotesingle{}}\NormalTok{),}
    \FunctionTok{c}\NormalTok{(}\StringTok{\textquotesingle{}rosa\textquotesingle{}}\NormalTok{, }\StringTok{\textquotesingle{}magenta\textquotesingle{}}\NormalTok{),}
    \ConstantTok{NA}\NormalTok{,}
    \StringTok{\textquotesingle{}branco\textquotesingle{}}
\NormalTok{  )}
\NormalTok{)}
\end{Highlighting}
\end{Shaded}

\begin{Shaded}
\begin{Highlighting}[]
\NormalTok{cores}
\end{Highlighting}
\end{Shaded}

\begin{verbatim}
# A tibble: 4 x 2
  pessoa `cor favorita`
  <chr>  <list>        
1 João   <chr [2]>     
2 Maria  <chr [2]>     
3 Pedro  <lgl [1]>     
4 Ana    <chr [1]>     
\end{verbatim}
\item
  Use \texttt{View()} para examinar interativamente o conteúdo de uma
  coluna-lista:

\begin{Shaded}
\begin{Highlighting}[]
\NormalTok{cores }\SpecialCharTok{\%\textgreater{}\%} \FunctionTok{View}\NormalTok{()}
\end{Highlighting}
\end{Shaded}
\end{itemize}

\section{\texorpdfstring{Operador de \emph{pipe}
(\texttt{\%\textgreater{}\%})}{Operador de pipe (\%\textgreater\%)}}\label{operador-de-pipe}

\begin{itemize}
\tightlist
\item
  O \texttt{tidyverse} inclui o pacote \texttt{magrittr}, que contém o
  operador \texttt{\%\textgreater{}\%}, chamado \emph{pipe}.\footnote{Por
    que o nome do pacote e o nome do operador formam um trocadilho?}
\end{itemize}

\begin{itemize}
\item
  A idéia é facilitar a leitura de {\hl{composições de funções}}. O
  código

  ::: \{.cell layout-align=``center''\}

\begin{Shaded}
\begin{Highlighting}[]
\NormalTok{y }\OtherTok{\textless{}{-}} \FunctionTok{h}\NormalTok{(}\FunctionTok{g}\NormalTok{(}\FunctionTok{f}\NormalTok{(x)))}
\end{Highlighting}
\end{Shaded}

  :::

  pode ser escrito como

  ::: \{.cell layout-align=``center''\}

\begin{Shaded}
\begin{Highlighting}[]
\NormalTok{y }\OtherTok{\textless{}{-}}\NormalTok{ x }\SpecialCharTok{\%\textgreater{}\%} \FunctionTok{f}\NormalTok{() }\SpecialCharTok{\%\textgreater{}\%} \FunctionTok{g}\NormalTok{() }\SpecialCharTok{\%\textgreater{}\%} \FunctionTok{h}\NormalTok{()}
\end{Highlighting}
\end{Shaded}

  :::
\item
  Esta segunda versão é mais fiel à ordem em que as operações acontecem.
\item
  Na verdade, R tem um operador de {\hl{atribuição para a direita}}, mas
  poucas pessoas recomendam usá-lo:

  ::: \{.cell layout-align=``center''\}

\begin{Shaded}
\begin{Highlighting}[]
\NormalTok{x }\SpecialCharTok{\%\textgreater{}\%} \FunctionTok{f}\NormalTok{() }\SpecialCharTok{\%\textgreater{}\%} \FunctionTok{g}\NormalTok{() }\SpecialCharTok{\%\textgreater{}\%} \FunctionTok{h}\NormalTok{() }\OtherTok{{-}\textgreater{}}\NormalTok{ y}
\end{Highlighting}
\end{Shaded}

  :::
\item
  Se \texttt{f}, \texttt{g} e \texttt{h} forem funções de um argumento
  só, os parênteses podem ser omitidos:

  ::: \{.cell layout-align=``center''\}

\begin{Shaded}
\begin{Highlighting}[]
\NormalTok{y }\OtherTok{\textless{}{-}}\NormalTok{ x }\SpecialCharTok{\%\textgreater{}\%}\NormalTok{ f }\SpecialCharTok{\%\textgreater{}\%}\NormalTok{ g }\SpecialCharTok{\%\textgreater{}\%}\NormalTok{ h}
\end{Highlighting}
\end{Shaded}

  :::
\item
  Se a função \texttt{f} tiver outros argumentos, escreva-os normalmente
  na chamada a \texttt{f}:

  ::: \{.cell layout-align=``center''\}

\begin{Shaded}
\begin{Highlighting}[]
\NormalTok{y }\OtherTok{\textless{}{-}}\NormalTok{ x }\SpecialCharTok{\%\textgreater{}\%} \FunctionTok{mean}\NormalTok{(}\AttributeTok{na.rm =} \ConstantTok{TRUE}\NormalTok{)}
\end{Highlighting}
\end{Shaded}

  :::
\item
  O \emph{pipe} \texttt{EXP\ \%\textgreater{}\%\ f(...)} sempre insere o
  resultado da expressão \texttt{EXP} como o {\hl{primeiro argumento da
  função {\mbox{\texttt{f}}}}}.
\item
  Se você precisar que o resultado da expressão \texttt{EXP} seja
  inserido em outra posição na lista de argumentos de \texttt{f}, use um
  ponto ``\texttt{.}'' para isso:

  ::: \{.cell layout-align=``center''\}

\begin{Shaded}
\begin{Highlighting}[]
\NormalTok{x }\SpecialCharTok{\%\textgreater{}\%} \FunctionTok{consultar}\NormalTok{(df, .)}
\end{Highlighting}
\end{Shaded}

  :::

  equivale a

  ::: \{.cell layout-align=``center''\}

\begin{Shaded}
\begin{Highlighting}[]
\FunctionTok{consultar}\NormalTok{(df, x)}
\end{Highlighting}
\end{Shaded}

  :::
\end{itemize}

\section{\texorpdfstring{Formato
\emph{tidy}}{Formato tidy}}\label{formato-tidy}

\begin{itemize}
\item
  Nossa última versão da \emph{tibble} \texttt{cores} é um pouco mais
  complexa do que deveria ser:

\begin{Shaded}
\begin{Highlighting}[]
\NormalTok{cores}
\end{Highlighting}
\end{Shaded}

\begin{verbatim}
# A tibble: 4 x 2
  pessoa `cor favorita`
  <chr>  <list>        
1 João   <chr [2]>     
2 Maria  <chr [2]>     
3 Pedro  <lgl [1]>     
4 Ana    <chr [1]>     
\end{verbatim}
\item
  O formato \emph{tidy} exige que

  \begin{enumerate}
  \def\labelenumi{\arabic{enumi}.}
  \item
    {\hl{Cada linha}} da \emph{tibble} corresponda a uma
    {\hl{observação}} sobre um indivíduo,
  \item
    {\hl{Cada coluna}} corresponda a {\hl{uma variável observada}}, e
  \item
    {\hl{Cada célula}} contenha {\hl{um valor}} da variável.
  \end{enumerate}
\item
  Na \emph{tibble} \texttt{cores}, a primeira e a segunda exigências são
  satisfeitas, mas a terceira não, pois algumas células contém valores
  múltiplos.
\item
  A \emph{tibble} não está no formato \emph{tidy}.
\item
  Podemos ``extrair'' estes vetores ``aninhados'' usando o comando
  \texttt{unnest}, do pacote \texttt{tidyr}:

  ::: \{.cell layout-align=``center''\}

\begin{Shaded}
\begin{Highlighting}[]
\NormalTok{cores }\OtherTok{\textless{}{-}}\NormalTok{ cores }\SpecialCharTok{\%\textgreater{}\%} 
  \FunctionTok{unnest}\NormalTok{(}\StringTok{\textasciigrave{}}\AttributeTok{cor favorita}\StringTok{\textasciigrave{}}\NormalTok{)}
\end{Highlighting}
\end{Shaded}

  :::

\begin{Shaded}
\begin{Highlighting}[]
\NormalTok{cores}
\end{Highlighting}
\end{Shaded}

\begin{verbatim}
# A tibble: 6 x 2
  pessoa `cor favorita`
  <chr>  <chr>         
1 João   azul          
2 João   roxo          
3 Maria  rosa          
4 Maria  magenta       
5 Pedro  <NA>          
6 Ana    branco        
\end{verbatim}
\item
  {\hl{A maioria das funções do {\mbox{\texttt{tidyverse}}} exige que as
  \emph{tibbles} estejam neste formato \emph{tidy}.}}
\item
  Um exemplo mais complexo é o \emph{dataset} \texttt{billboard}, com as
  seguintes colunas (para cada música que estava no \emph{top 100} da
  Billboard no ano de $2000$):

  \begin{itemize}
  \item
    Nome do artista ou banda;
  \item
    Nome da música;
  \item
    Data em que a música entrou no \emph{top 100} da Billboard;
  \item
    Para cada uma das $76$ semanas seguintes, a posição da música no
    \emph{top 100}.

\begin{Shaded}
\begin{Highlighting}[]
\NormalTok{billboard }\SpecialCharTok{\%\textgreater{}\%} \FunctionTok{glimpse}\NormalTok{()}
\end{Highlighting}
\end{Shaded}

\begin{verbatim}
Rows: 317
Columns: 79
$ artist       <chr> "2 Pac", "2Ge+her", "3 Doors Down", "3 Doors D~
$ track        <chr> "Baby Don't Cry (Keep...", "The Hardest Part O~
$ date.entered <date> 2000-02-26, 2000-09-02, 2000-04-08, 2000-10-2~
$ wk1          <dbl> 87, 91, 81, 76, 57, 51, 97, 84, 59, 76, 84, 57~
$ wk2          <dbl> 82, 87, 70, 76, 34, 39, 97, 62, 53, 76, 84, 47~
$ wk3          <dbl> 72, 92, 68, 72, 25, 34, 96, 51, 38, 74, 75, 45~
$ wk4          <dbl> 77, NA, 67, 69, 17, 26, 95, 41, 28, 69, 73, 29~
$ wk5          <dbl> 87, NA, 66, 67, 17, 26, 100, 38, 21, 68, 73, 2~
$ wk6          <dbl> 94, NA, 57, 65, 31, 19, NA, 35, 18, 67, 69, 18~
$ wk7          <dbl> 99, NA, 54, 55, 36, 2, NA, 35, 16, 61, 68, 11,~
$ wk8          <dbl> NA, NA, 53, 59, 49, 2, NA, 38, 14, 58, 65, 9, ~
$ wk9          <dbl> NA, NA, 51, 62, 53, 3, NA, 38, 12, 57, 73, 9, ~
$ wk10         <dbl> NA, NA, 51, 61, 57, 6, NA, 36, 10, 59, 83, 11,~
$ wk11         <dbl> NA, NA, 51, 61, 64, 7, NA, 37, 9, 66, 92, 1, 1~
$ wk12         <dbl> NA, NA, 51, 59, 70, 22, NA, 37, 8, 68, NA, 1, ~
$ wk13         <dbl> NA, NA, 47, 61, 75, 29, NA, 38, 6, 61, NA, 1, ~
$ wk14         <dbl> NA, NA, 44, 66, 76, 36, NA, 49, 1, 67, NA, 1, ~
$ wk15         <dbl> NA, NA, 38, 72, 78, 47, NA, 61, 2, 59, NA, 4, ~
$ wk16         <dbl> NA, NA, 28, 76, 85, 67, NA, 63, 2, 63, NA, 8, ~
$ wk17         <dbl> NA, NA, 22, 75, 92, 66, NA, 62, 2, 67, NA, 12,~
$ wk18         <dbl> NA, NA, 18, 67, 96, 84, NA, 67, 2, 71, NA, 22,~
$ wk19         <dbl> NA, NA, 18, 73, NA, 93, NA, 83, 3, 79, NA, 23,~
$ wk20         <dbl> NA, NA, 14, 70, NA, 94, NA, 86, 4, 89, NA, 43,~
$ wk21         <dbl> NA, NA, 12, NA, NA, NA, NA, NA, 5, NA, NA, 44,~
$ wk22         <dbl> NA, NA, 7, NA, NA, NA, NA, NA, 5, NA, NA, NA, ~
$ wk23         <dbl> NA, NA, 6, NA, NA, NA, NA, NA, 6, NA, NA, NA, ~
$ wk24         <dbl> NA, NA, 6, NA, NA, NA, NA, NA, 9, NA, NA, NA, ~
$ wk25         <dbl> NA, NA, 6, NA, NA, NA, NA, NA, 13, NA, NA, NA,~
$ wk26         <dbl> NA, NA, 5, NA, NA, NA, NA, NA, 14, NA, NA, NA,~
$ wk27         <dbl> NA, NA, 5, NA, NA, NA, NA, NA, 16, NA, NA, NA,~
$ wk28         <dbl> NA, NA, 4, NA, NA, NA, NA, NA, 23, NA, NA, NA,~
$ wk29         <dbl> NA, NA, 4, NA, NA, NA, NA, NA, 22, NA, NA, NA,~
$ wk30         <dbl> NA, NA, 4, NA, NA, NA, NA, NA, 33, NA, NA, NA,~
$ wk31         <dbl> NA, NA, 4, NA, NA, NA, NA, NA, 36, NA, NA, NA,~
$ wk32         <dbl> NA, NA, 3, NA, NA, NA, NA, NA, 43, NA, NA, NA,~
$ wk33         <dbl> NA, NA, 3, NA, NA, NA, NA, NA, NA, NA, NA, NA,~
$ wk34         <dbl> NA, NA, 3, NA, NA, NA, NA, NA, NA, NA, NA, NA,~
$ wk35         <dbl> NA, NA, 4, NA, NA, NA, NA, NA, NA, NA, NA, NA,~
$ wk36         <dbl> NA, NA, 5, NA, NA, NA, NA, NA, NA, NA, NA, NA,~
$ wk37         <dbl> NA, NA, 5, NA, NA, NA, NA, NA, NA, NA, NA, NA,~
$ wk38         <dbl> NA, NA, 9, NA, NA, NA, NA, NA, NA, NA, NA, NA,~
$ wk39         <dbl> NA, NA, 9, NA, NA, NA, NA, NA, NA, NA, NA, NA,~
$ wk40         <dbl> NA, NA, 15, NA, NA, NA, NA, NA, NA, NA, NA, NA~
$ wk41         <dbl> NA, NA, 14, NA, NA, NA, NA, NA, NA, NA, NA, NA~
$ wk42         <dbl> NA, NA, 13, NA, NA, NA, NA, NA, NA, NA, NA, NA~
$ wk43         <dbl> NA, NA, 14, NA, NA, NA, NA, NA, NA, NA, NA, NA~
$ wk44         <dbl> NA, NA, 16, NA, NA, NA, NA, NA, NA, NA, NA, NA~
$ wk45         <dbl> NA, NA, 17, NA, NA, NA, NA, NA, NA, NA, NA, NA~
$ wk46         <dbl> NA, NA, 21, NA, NA, NA, NA, NA, NA, NA, NA, NA~
$ wk47         <dbl> NA, NA, 22, NA, NA, NA, NA, NA, NA, NA, NA, NA~
$ wk48         <dbl> NA, NA, 24, NA, NA, NA, NA, NA, NA, NA, NA, NA~
$ wk49         <dbl> NA, NA, 28, NA, NA, NA, NA, NA, NA, NA, NA, NA~
$ wk50         <dbl> NA, NA, 33, NA, NA, NA, NA, NA, NA, NA, NA, NA~
$ wk51         <dbl> NA, NA, 42, NA, NA, NA, NA, NA, NA, NA, NA, NA~
$ wk52         <dbl> NA, NA, 42, NA, NA, NA, NA, NA, NA, NA, NA, NA~
$ wk53         <dbl> NA, NA, 49, NA, NA, NA, NA, NA, NA, NA, NA, NA~
$ wk54         <dbl> NA, NA, NA, NA, NA, NA, NA, NA, NA, NA, NA, NA~
$ wk55         <dbl> NA, NA, NA, NA, NA, NA, NA, NA, NA, NA, NA, NA~
$ wk56         <dbl> NA, NA, NA, NA, NA, NA, NA, NA, NA, NA, NA, NA~
$ wk57         <dbl> NA, NA, NA, NA, NA, NA, NA, NA, NA, NA, NA, NA~
$ wk58         <dbl> NA, NA, NA, NA, NA, NA, NA, NA, NA, NA, NA, NA~
$ wk59         <dbl> NA, NA, NA, NA, NA, NA, NA, NA, NA, NA, NA, NA~
$ wk60         <dbl> NA, NA, NA, NA, NA, NA, NA, NA, NA, NA, NA, NA~
$ wk61         <dbl> NA, NA, NA, NA, NA, NA, NA, NA, NA, NA, NA, NA~
$ wk62         <dbl> NA, NA, NA, NA, NA, NA, NA, NA, NA, NA, NA, NA~
$ wk63         <dbl> NA, NA, NA, NA, NA, NA, NA, NA, NA, NA, NA, NA~
$ wk64         <dbl> NA, NA, NA, NA, NA, NA, NA, NA, NA, NA, NA, NA~
$ wk65         <dbl> NA, NA, NA, NA, NA, NA, NA, NA, NA, NA, NA, NA~
$ wk66         <lgl> NA, NA, NA, NA, NA, NA, NA, NA, NA, NA, NA, NA~
$ wk67         <lgl> NA, NA, NA, NA, NA, NA, NA, NA, NA, NA, NA, NA~
$ wk68         <lgl> NA, NA, NA, NA, NA, NA, NA, NA, NA, NA, NA, NA~
$ wk69         <lgl> NA, NA, NA, NA, NA, NA, NA, NA, NA, NA, NA, NA~
$ wk70         <lgl> NA, NA, NA, NA, NA, NA, NA, NA, NA, NA, NA, NA~
$ wk71         <lgl> NA, NA, NA, NA, NA, NA, NA, NA, NA, NA, NA, NA~
$ wk72         <lgl> NA, NA, NA, NA, NA, NA, NA, NA, NA, NA, NA, NA~
$ wk73         <lgl> NA, NA, NA, NA, NA, NA, NA, NA, NA, NA, NA, NA~
$ wk74         <lgl> NA, NA, NA, NA, NA, NA, NA, NA, NA, NA, NA, NA~
$ wk75         <lgl> NA, NA, NA, NA, NA, NA, NA, NA, NA, NA, NA, NA~
$ wk76         <lgl> NA, NA, NA, NA, NA, NA, NA, NA, NA, NA, NA, NA~
\end{verbatim}
  \end{itemize}
\item
  Vamos renomear as colunas:

  ::: \{.cell layout-align=``center''\}

\begin{Shaded}
\begin{Highlighting}[]
\NormalTok{bb }\OtherTok{\textless{}{-}}\NormalTok{ billboard }\SpecialCharTok{\%\textgreater{}\%} 
  \FunctionTok{rename}\NormalTok{(}
    \AttributeTok{artista =}\NormalTok{ artist,}
    \AttributeTok{musica =}\NormalTok{ track,}
    \AttributeTok{entrou =}\NormalTok{ date.entered}
\NormalTok{  )}
\end{Highlighting}
\end{Shaded}

  :::

\begin{Shaded}
\begin{Highlighting}[]
\NormalTok{bb }\SpecialCharTok{\%\textgreater{}\%} \FunctionTok{head}\NormalTok{()}
\end{Highlighting}
\end{Shaded}

\begin{verbatim}
# A tibble: 6 x 79
  artista musica entrou       wk1   wk2   wk3   wk4   wk5   wk6   wk7
  <chr>   <chr>  <date>     <dbl> <dbl> <dbl> <dbl> <dbl> <dbl> <dbl>
1 2 Pac   Baby ~ 2000-02-26    87    82    72    77    87    94    99
2 2Ge+her The H~ 2000-09-02    91    87    92    NA    NA    NA    NA
3 3 Door~ Krypt~ 2000-04-08    81    70    68    67    66    57    54
4 3 Door~ Loser  2000-10-21    76    76    72    69    67    65    55
5 504 Bo~ Wobbl~ 2000-04-15    57    34    25    17    17    31    36
6 98^0    Give ~ 2000-08-19    51    39    34    26    26    19     2
# i 69 more variables: wk8 <dbl>, wk9 <dbl>, wk10 <dbl>, wk11 <dbl>,
#   wk12 <dbl>, wk13 <dbl>, wk14 <dbl>, wk15 <dbl>, wk16 <dbl>,
#   wk17 <dbl>, wk18 <dbl>, wk19 <dbl>, wk20 <dbl>, wk21 <dbl>,
#   wk22 <dbl>, wk23 <dbl>, wk24 <dbl>, wk25 <dbl>, wk26 <dbl>,
#   wk27 <dbl>, wk28 <dbl>, wk29 <dbl>, wk30 <dbl>, wk31 <dbl>,
#   wk32 <dbl>, wk33 <dbl>, wk34 <dbl>, wk35 <dbl>, wk36 <dbl>,
#   wk37 <dbl>, wk38 <dbl>, wk39 <dbl>, wk40 <dbl>, wk41 <dbl>, ...
\end{verbatim}
\item
  {\hl{O que é uma observação}} neste conjunto de dados?

  {\hl{A posição, em uma semana, de uma música}} que esteve no
  \emph{top} $100$ da \emph{Billboard} durante o ano \emph{2000}.
\item
  {\hl{Quais são as variáveis}} que qualificam cada observação?

  \begin{itemize}
  \item
    O artista,
  \item
    O título da música,
  \item
    A posiçao da música no \emph{top} $100$ da \emph{Billboard} em cada
    uma das $76$ semanas depois que ela entrou na lista.
  \end{itemize}
\item
  Este último item é complexo, e o criador da \emph{tibble} decidiu
  criar uma coluna por semana.
\item
  {\hl{Uma decisão ruim, pois existe informação embutida nos nomes das
  colunas.}} A coluna \texttt{wk68} corresponde à posição da música na
  semana $68$ após ela entrar na lista, {\hl{mas o número da semana só
  aparece no nome da coluna}}!
\item
  Isto {\hl{nunca}} deve acontecer. {\hl{A informação deve sempre estar
  nas células.}}
\item
  Vamos simplificar as coisas criando duas colunas:

  \begin{itemize}
  \item
    \texttt{semana}, com o número da semana; perceba que esta informação
    vem dos nomes das colunas,
  \item
    \texttt{pos}, com a posição da música naquela semana; esta
    informação vem das células.
  \end{itemize}
\item
  A \emph{tibble}, que antes era larga, {\hl{vai ser mais estreita e
  mais longa}}.
\item
  A função \texttt{pivot\_longer}, do pacote \texttt{tidyr}, vai fazer o
  trabalho --- inclusive extraindo os números das semanas dos nomes das
  colunas:

  ::: \{.cell layout-align=``center''\}

\begin{Shaded}
\begin{Highlighting}[]
\NormalTok{bb\_tidy }\OtherTok{\textless{}{-}}\NormalTok{ bb }\SpecialCharTok{\%\textgreater{}\%} 
  \FunctionTok{pivot\_longer}\NormalTok{(}
\NormalTok{    wk1}\SpecialCharTok{:}\NormalTok{wk76,}
    \AttributeTok{names\_to =} \StringTok{\textquotesingle{}semana\textquotesingle{}}\NormalTok{,}
    \AttributeTok{names\_prefix =} \StringTok{\textquotesingle{}wk\textquotesingle{}}\NormalTok{,}
    \AttributeTok{names\_transform =} \FunctionTok{list}\NormalTok{(}
      \AttributeTok{semana =}\NormalTok{ as.integer}
\NormalTok{    ),}
    \AttributeTok{values\_to =} \StringTok{\textquotesingle{}pos\textquotesingle{}}
\NormalTok{  )}
\end{Highlighting}
\end{Shaded}

  :::

\begin{Shaded}
\begin{Highlighting}[]
\NormalTok{bb\_tidy}
\end{Highlighting}
\end{Shaded}

\begin{verbatim}
# A tibble: 24.092 x 5
  artista musica                  entrou     semana   pos
  <chr>   <chr>                   <date>      <int> <dbl>
1 2 Pac   Baby Don't Cry (Keep... 2000-02-26      1    87
2 2 Pac   Baby Don't Cry (Keep... 2000-02-26      2    82
3 2 Pac   Baby Don't Cry (Keep... 2000-02-26      3    72
4 2 Pac   Baby Don't Cry (Keep... 2000-02-26      4    77
5 2 Pac   Baby Don't Cry (Keep... 2000-02-26      5    87
6 2 Pac   Baby Don't Cry (Keep... 2000-02-26      6    94
# i 24.086 more rows
\end{verbatim}
\item
  O R só mostra, por \emph{default}, as $1000$ primeiras linhas de uma
  \emph{tibble}.
\item
  Na verdade, o número de linhas da tabela original era

  ::: \{.cell layout-align=``center''\}

\begin{Shaded}
\begin{Highlighting}[]
\NormalTok{bb }\SpecialCharTok{\%\textgreater{}\%} \FunctionTok{nrow}\NormalTok{()}
\end{Highlighting}
\end{Shaded}

  ::: \{.cell-output .cell-output-stdout\}

\begin{verbatim}
[1] 317
\end{verbatim}

  ::: :::
\item
  O número de linhas, depois de \texttt{pivot\_longer}, ficou:

  ::: \{.cell layout-align=``center''\}

\begin{Shaded}
\begin{Highlighting}[]
\NormalTok{bb\_tidy }\SpecialCharTok{\%\textgreater{}\%} \FunctionTok{nrow}\NormalTok{()}
\end{Highlighting}
\end{Shaded}

  ::: \{.cell-output .cell-output-stdout\}

\begin{verbatim}
[1] 24092
\end{verbatim}

  ::: :::
\item
  Existem linhas onde \texttt{pos} tem o valor \texttt{NA}. São
  resultado da organização original dos dados, onde o \texttt{NA}
  indicava que a música não estava no \emph{top} $100$ naquela semana.
\item
  No novo formato, a ausência da linha com aquele número de semana já
  basta para indicar isto. Então, vamos eliminar as linhas onde
  \texttt{pos} é \texttt{NA}.
\item
  A função \texttt{filter} {\hl{mantém}} as linhas que {\hl{satisfazem}}
  a condição dada; por isso, a condição é ``\texttt{pos} não é
  \texttt{NA}'':

  ::: \{.cell layout-align=``center''\}

\begin{Shaded}
\begin{Highlighting}[]
\NormalTok{bb\_tidy }\OtherTok{\textless{}{-}}\NormalTok{ bb\_tidy }\SpecialCharTok{\%\textgreater{}\%} 
  \FunctionTok{filter}\NormalTok{(}\SpecialCharTok{!}\FunctionTok{is.na}\NormalTok{(pos))}
\end{Highlighting}
\end{Shaded}

  :::

\begin{Shaded}
\begin{Highlighting}[]
\NormalTok{bb\_tidy}
\end{Highlighting}
\end{Shaded}

\begin{verbatim}
# A tibble: 5.307 x 5
  artista musica                  entrou     semana   pos
  <chr>   <chr>                   <date>      <int> <dbl>
1 2 Pac   Baby Don't Cry (Keep... 2000-02-26      1    87
2 2 Pac   Baby Don't Cry (Keep... 2000-02-26      2    82
3 2 Pac   Baby Don't Cry (Keep... 2000-02-26      3    72
4 2 Pac   Baby Don't Cry (Keep... 2000-02-26      4    77
5 2 Pac   Baby Don't Cry (Keep... 2000-02-26      5    87
6 2 Pac   Baby Don't Cry (Keep... 2000-02-26      6    94
# i 5.301 more rows
\end{verbatim}
\item
  O número de linhas ficou

  ::: \{.cell layout-align=``center''\}

\begin{Shaded}
\begin{Highlighting}[]
\NormalTok{bb\_tidy }\SpecialCharTok{\%\textgreater{}\%} \FunctionTok{nrow}\NormalTok{()}
\end{Highlighting}
\end{Shaded}

  ::: \{.cell-output .cell-output-stdout\}

\begin{verbatim}
[1] 5307
\end{verbatim}

  ::: :::
\end{itemize}

\subsection{Exercícios}\label{exercuxedcios}

\begin{itemize}
\item
  Todas as semanas deste conjunto de dados são do ano $2000$?
\item
  Qual é o tipo do {\hl{primeiro}} argumento da função
  \texttt{filter()}?
\end{itemize}

\section{Manipulando os dados}\label{manipulando-os-dados}

\subsection{\texorpdfstring{Criando novas colunas: \texttt{mutate},
\texttt{transmute}}{Criando novas colunas: mutate, transmute}}\label{criando-novas-colunas-mutate-transmute}

\begin{itemize}
\item
  O \emph{data frame}\footnote{Considere \emph{data frame} como sinônimo
    de \emph{tibble}. Na verdade, \emph{tibbles} formam um superconjunto
    de \emph{data frames}: todo \emph{data frame} é uma \emph{tibble},
    mas nem toda \emph{tibble} é um \emph{data frame}.} \texttt{cars}
  tem dados (de $1920$!) sobre as distâncias de frenagem (em pés) de um
  carro viajando a diversas velocidades (em milhas por hora):

\begin{Shaded}
\begin{Highlighting}[]
\NormalTok{cars}
\end{Highlighting}
\end{Shaded}

\begin{verbatim}
   speed dist
1      4    2
2      4   10
3      7    4
4      7   22
5      8   16
6      9   10
7     10   18
8     10   26
9     10   34
10    11   17
11    11   28
12    12   14
13    12   20
14    12   24
15    12   28
16    13   26
17    13   34
18    13   34
19    13   46
20    14   26
21    14   36
22    14   60
23    14   80
24    15   20
25    15   26
26    15   54
27    16   32
28    16   40
29    17   32
30    17   40
31    17   50
32    18   42
33    18   56
34    18   76
35    18   84
36    19   36
37    19   46
38    19   68
39    20   32
40    20   48
41    20   52
42    20   56
43    20   64
44    22   66
45    23   54
46    24   70
47    24   92
48    24   93
49    24  120
50    25   85
\end{verbatim}
\item
  Vamos criar colunas novas com os valores convertidos para km/h e
  metros; além disso, uma coluna com a taxa de frenagem:

\begin{Shaded}
\begin{Highlighting}[]
\NormalTok{cars }\SpecialCharTok{\%\textgreater{}\%} 
  \FunctionTok{mutate}\NormalTok{(}
    \AttributeTok{velocidade =}\NormalTok{ speed }\SpecialCharTok{*} \FloatTok{1.6}\NormalTok{,}
    \AttributeTok{distancia =}\NormalTok{ dist }\SpecialCharTok{*}\NormalTok{ .}\DecValTok{33}\NormalTok{,}
    \AttributeTok{taxa =}\NormalTok{ velocidade }\SpecialCharTok{/}\NormalTok{ distancia}
\NormalTok{  )}
\end{Highlighting}
\end{Shaded}

\begin{verbatim}
   speed dist velocidade distancia      taxa
1      4    2        6,4      0,66 9,6969697
2      4   10        6,4      3,30 1,9393939
3      7    4       11,2      1,32 8,4848485
4      7   22       11,2      7,26 1,5426997
5      8   16       12,8      5,28 2,4242424
6      9   10       14,4      3,30 4,3636364
7     10   18       16,0      5,94 2,6936027
8     10   26       16,0      8,58 1,8648019
9     10   34       16,0     11,22 1,4260250
10    11   17       17,6      5,61 3,1372549
11    11   28       17,6      9,24 1,9047619
12    12   14       19,2      4,62 4,1558442
13    12   20       19,2      6,60 2,9090909
14    12   24       19,2      7,92 2,4242424
15    12   28       19,2      9,24 2,0779221
16    13   26       20,8      8,58 2,4242424
17    13   34       20,8     11,22 1,8538324
18    13   34       20,8     11,22 1,8538324
19    13   46       20,8     15,18 1,3702240
20    14   26       22,4      8,58 2,6107226
21    14   36       22,4     11,88 1,8855219
22    14   60       22,4     19,80 1,1313131
23    14   80       22,4     26,40 0,8484848
24    15   20       24,0      6,60 3,6363636
25    15   26       24,0      8,58 2,7972028
26    15   54       24,0     17,82 1,3468013
27    16   32       25,6     10,56 2,4242424
28    16   40       25,6     13,20 1,9393939
29    17   32       27,2     10,56 2,5757576
30    17   40       27,2     13,20 2,0606061
31    17   50       27,2     16,50 1,6484848
32    18   42       28,8     13,86 2,0779221
33    18   56       28,8     18,48 1,5584416
34    18   76       28,8     25,08 1,1483254
35    18   84       28,8     27,72 1,0389610
36    19   36       30,4     11,88 2,5589226
37    19   46       30,4     15,18 2,0026350
38    19   68       30,4     22,44 1,3547237
39    20   32       32,0     10,56 3,0303030
40    20   48       32,0     15,84 2,0202020
41    20   52       32,0     17,16 1,8648019
42    20   56       32,0     18,48 1,7316017
43    20   64       32,0     21,12 1,5151515
44    22   66       35,2     21,78 1,6161616
45    23   54       36,8     17,82 2,0650954
46    24   70       38,4     23,10 1,6623377
47    24   92       38,4     30,36 1,2648221
48    24   93       38,4     30,69 1,2512219
49    24  120       38,4     39,60 0,9696970
50    25   85       40,0     28,05 1,4260250
\end{verbatim}
\item
  Perceba que as colunas antigas continuam lá. {\hl{Se quiser manter
  apenas as colunas novas, use {\mbox{\texttt{transmute}}}}}:

\begin{Shaded}
\begin{Highlighting}[]
\NormalTok{cars }\SpecialCharTok{\%\textgreater{}\%} 
  \FunctionTok{transmute}\NormalTok{(}
    \AttributeTok{velocidade =}\NormalTok{ speed }\SpecialCharTok{*} \FloatTok{1.6}\NormalTok{,}
    \AttributeTok{distancia =}\NormalTok{ dist }\SpecialCharTok{*}\NormalTok{ .}\DecValTok{33}\NormalTok{,}
    \AttributeTok{taxa =}\NormalTok{ velocidade }\SpecialCharTok{/}\NormalTok{ distancia}
\NormalTok{  )}
\end{Highlighting}
\end{Shaded}

\begin{verbatim}
   velocidade distancia      taxa
1         6,4      0,66 9,6969697
2         6,4      3,30 1,9393939
3        11,2      1,32 8,4848485
4        11,2      7,26 1,5426997
5        12,8      5,28 2,4242424
6        14,4      3,30 4,3636364
7        16,0      5,94 2,6936027
8        16,0      8,58 1,8648019
9        16,0     11,22 1,4260250
10       17,6      5,61 3,1372549
11       17,6      9,24 1,9047619
12       19,2      4,62 4,1558442
13       19,2      6,60 2,9090909
14       19,2      7,92 2,4242424
15       19,2      9,24 2,0779221
16       20,8      8,58 2,4242424
17       20,8     11,22 1,8538324
18       20,8     11,22 1,8538324
19       20,8     15,18 1,3702240
20       22,4      8,58 2,6107226
21       22,4     11,88 1,8855219
22       22,4     19,80 1,1313131
23       22,4     26,40 0,8484848
24       24,0      6,60 3,6363636
25       24,0      8,58 2,7972028
26       24,0     17,82 1,3468013
27       25,6     10,56 2,4242424
28       25,6     13,20 1,9393939
29       27,2     10,56 2,5757576
30       27,2     13,20 2,0606061
31       27,2     16,50 1,6484848
32       28,8     13,86 2,0779221
33       28,8     18,48 1,5584416
34       28,8     25,08 1,1483254
35       28,8     27,72 1,0389610
36       30,4     11,88 2,5589226
37       30,4     15,18 2,0026350
38       30,4     22,44 1,3547237
39       32,0     10,56 3,0303030
40       32,0     15,84 2,0202020
41       32,0     17,16 1,8648019
42       32,0     18,48 1,7316017
43       32,0     21,12 1,5151515
44       35,2     21,78 1,6161616
45       36,8     17,82 2,0650954
46       38,4     23,10 1,6623377
47       38,4     30,36 1,2648221
48       38,4     30,69 1,2512219
49       38,4     39,60 0,9696970
50       40,0     28,05 1,4260250
\end{verbatim}
\item
  Ou use o argumento \texttt{.keep} de \texttt{mutate} para escolher com
  mais precisão. Veja a ajuda de \texttt{mutate}.
\end{itemize}

\subsection{\texorpdfstring{Selecionando colunas: \texttt{select},
\texttt{distinct},
\texttt{pull}}{Selecionando colunas: select, distinct, pull}}\label{selecionando-colunas-select-distinct-pull}

\begin{itemize}
\item
  Vamos voltar à nossa \emph{tibble} dos \emph{top} $100$ da
  \emph{Billboard}.
\item
  Para ver só a coluna de artistas:

\begin{Shaded}
\begin{Highlighting}[]
\NormalTok{bb\_tidy }\SpecialCharTok{\%\textgreater{}\%} 
  \FunctionTok{select}\NormalTok{(artista)}
\end{Highlighting}
\end{Shaded}

\begin{verbatim}
# A tibble: 5.307 x 1
  artista
  <chr>  
1 2 Pac  
2 2 Pac  
3 2 Pac  
4 2 Pac  
5 2 Pac  
6 2 Pac  
# i 5.301 more rows
\end{verbatim}
\item
  Para eliminar as repetições:

\begin{Shaded}
\begin{Highlighting}[]
\NormalTok{bb\_tidy }\SpecialCharTok{\%\textgreater{}\%} 
  \FunctionTok{select}\NormalTok{(artista) }\SpecialCharTok{\%\textgreater{}\%} 
  \FunctionTok{distinct}\NormalTok{()}
\end{Highlighting}
\end{Shaded}

\begin{verbatim}
# A tibble: 228 x 1
  artista     
  <chr>       
1 2 Pac       
2 2Ge+her     
3 3 Doors Down
4 504 Boyz    
5 98^0        
6 A*Teens     
# i 222 more rows
\end{verbatim}
\item
  Para ver artistas e músicas:

\begin{Shaded}
\begin{Highlighting}[]
\NormalTok{bb\_tidy }\SpecialCharTok{\%\textgreater{}\%} 
  \FunctionTok{select}\NormalTok{(artista, musica) }\SpecialCharTok{\%\textgreater{}\%} 
  \FunctionTok{distinct}\NormalTok{()}
\end{Highlighting}
\end{Shaded}

\begin{verbatim}
# A tibble: 317 x 2
  artista      musica                 
  <chr>        <chr>                  
1 2 Pac        Baby Don't Cry (Keep...
2 2Ge+her      The Hardest Part Of ...
3 3 Doors Down Kryptonite             
4 3 Doors Down Loser                  
5 504 Boyz     Wobble Wobble          
6 98^0         Give Me Just One Nig...
# i 311 more rows
\end{verbatim}
\item
  Para especificar colunas {\hl{a não mostrar}}, use o sinal de menos
  ``\texttt{-}'':

\begin{Shaded}
\begin{Highlighting}[]
\NormalTok{bb\_tidy }\SpecialCharTok{\%\textgreater{}\%} 
  \FunctionTok{select}\NormalTok{(}\SpecialCharTok{{-}}\FunctionTok{c}\NormalTok{(entrou, semana, pos))}
\end{Highlighting}
\end{Shaded}

\begin{verbatim}
# A tibble: 5.307 x 2
  artista musica                 
  <chr>   <chr>                  
1 2 Pac   Baby Don't Cry (Keep...
2 2 Pac   Baby Don't Cry (Keep...
3 2 Pac   Baby Don't Cry (Keep...
4 2 Pac   Baby Don't Cry (Keep...
5 2 Pac   Baby Don't Cry (Keep...
6 2 Pac   Baby Don't Cry (Keep...
# i 5.301 more rows
\end{verbatim}
\item
  Para {\hl{extrair uma coluna na forma de vetor}} (\texttt{unique} é
  uma função do R base, aplicável a vetores):

\begin{Shaded}
\begin{Highlighting}[]
\NormalTok{bb\_tidy }\SpecialCharTok{\%\textgreater{}\%} 
  \FunctionTok{pull}\NormalTok{(artista) }\SpecialCharTok{\%\textgreater{}\%} 
  \FunctionTok{unique}\NormalTok{()}
\end{Highlighting}
\end{Shaded}

\begin{verbatim}
  [1] "2 Pac"                         
  [2] "2Ge+her"                       
  [3] "3 Doors Down"                  
  [4] "504 Boyz"                      
  [5] "98^0"                          
  [6] "A*Teens"                       
  [7] "Aaliyah"                       
  [8] "Adams, Yolanda"                
  [9] "Adkins, Trace"                 
 [10] "Aguilera, Christina"           
 [11] "Alice Deejay"                  
 [12] "Allan, Gary"                   
 [13] "Amber"                         
 [14] "Anastacia"                     
 [15] "Anthony, Marc"                 
 [16] "Avant"                         
 [17] "BBMak"                         
 [18] "Backstreet Boys, The"          
 [19] "Badu, Erkyah"                  
 [20] "Baha Men"                      
 [21] "Barenaked Ladies"              
 [22] "Beenie Man"                    
 [23] "Before Dark"                   
 [24] "Bega, Lou"                     
 [25] "Big Punisher"                  
 [26] "Black Rob"                     
 [27] "Black, Clint"                  
 [28] "Blaque"                        
 [29] "Blige, Mary J."                
 [30] "Blink-182"                     
 [31] "Bloodhound Gang"               
 [32] "Bon Jovi"                      
 [33] "Braxton, Toni"                 
 [34] "Brock, Chad"                   
 [35] "Brooks & Dunn"                 
 [36] "Brooks, Garth"                 
 [37] "Byrd, Tracy"                   
 [38] "Cagle, Chris"                  
 [39] "Cam'ron"                       
 [40] "Carey, Mariah"                 
 [41] "Carter, Aaron"                 
 [42] "Carter, Torrey"                
 [43] "Changing Faces"                
 [44] "Chesney, Kenny"                
 [45] "Clark Family Experience"       
 [46] "Clark, Terri"                  
 [47] "Common"                        
 [48] "Counting Crows"                
 [49] "Creed"                         
 [50] "Cyrus, Billy Ray"              
 [51] "D'Angelo"                      
 [52] "DMX"                           
 [53] "Da Brat"                       
 [54] "Davidson, Clay"                
 [55] "De La Soul"                    
 [56] "Destiny's Child"               
 [57] "Diffie, Joe"                   
 [58] "Dion, Celine"                  
 [59] "Dixie Chicks, The"             
 [60] "Dr. Dre"                       
 [61] "Drama"                         
 [62] "Dream"                         
 [63] "Eastsidaz, The"                
 [64] "Eiffel 65"                     
 [65] "Elliott, Missy \"Misdemeanor\""
 [66] "Eminem"                        
 [67] "En Vogue"                      
 [68] "Estefan, Gloria"               
 [69] "Evans, Sara"                   
 [70] "Eve"                           
 [71] "Everclear"                     
 [72] "Fabian, Lara"                  
 [73] "Fatboy Slim"                   
 [74] "Filter"                        
 [75] "Foo Fighters"                  
 [76] "Fragma"                        
 [77] "Funkmaster Flex"               
 [78] "Ghostface Killah"              
 [79] "Gill, Vince"                   
 [80] "Gilman, Billy"                 
 [81] "Ginuwine"                      
 [82] "Goo Goo Dolls"                 
 [83] "Gray, Macy"                    
 [84] "Griggs, Andy"                  
 [85] "Guy"                           
 [86] "Hanson"                        
 [87] "Hart, Beth"                    
 [88] "Heatherly, Eric"               
 [89] "Henley, Don"                   
 [90] "Herndon, Ty"                   
 [91] "Hill, Faith"                   
 [92] "Hoku"                          
 [93] "Hollister, Dave"               
 [94] "Hot Boys"                      
 [95] "Houston, Whitney"              
 [96] "IMx"                           
 [97] "Ice Cube"                      
 [98] "Ideal"                         
 [99] "Iglesias, Enrique"             
[100] "J-Shin"                        
[101] "Ja Rule"                       
[102] "Jackson, Alan"                 
[103] "Jagged Edge"                   
[104] "Janet"                         
[105] "Jay-Z"                         
[106] "Jean, Wyclef"                  
[107] "Joe"                           
[108] "John, Elton"                   
[109] "Jones, Donell"                 
[110] "Jordan, Montell"               
[111] "Juvenile"                      
[112] "Kandi"                         
[113] "Keith, Toby"                   
[114] "Kelis"                         
[115] "Kenny G"                       
[116] "Kid Rock"                      
[117] "Kravitz, Lenny"                
[118] "Kumbia Kings"                  
[119] "LFO"                           
[120] "LL Cool J"                     
[121] "Larrieux, Amel"                
[122] "Lawrence, Tracy"               
[123] "Levert, Gerald"                
[124] "Lil Bow Wow"                   
[125] "Lil Wayne"                     
[126] "Lil' Kim"                      
[127] "Lil' Mo"                       
[128] "Lil' Zane"                     
[129] "Limp Bizkit"                   
[130] "Lonestar"                      
[131] "Lopez, Jennifer"               
[132] "Loveless, Patty"               
[133] "Lox"                           
[134] "Lucy Pearl"                    
[135] "Ludacris"                      
[136] "M2M"                           
[137] "Madison Avenue"                
[138] "Madonna"                       
[139] "Martin, Ricky"                 
[140] "Mary Mary"                     
[141] "Master P"                      
[142] "McBride, Martina"              
[143] "McEntire, Reba"                
[144] "McGraw, Tim"                   
[145] "McKnight, Brian"               
[146] "Messina, Jo Dee"               
[147] "Metallica"                     
[148] "Montgomery Gentry"             
[149] "Montgomery, John Michael"      
[150] "Moore, Chante"                 
[151] "Moore, Mandy"                  
[152] "Mumba, Samantha"               
[153] "Musiq"                         
[154] "Mya"                           
[155] "Mystikal"                      
[156] "N'Sync"                        
[157] "Nas"                           
[158] "Nelly"                         
[159] "Next"                          
[160] "Nine Days"                     
[161] "No Doubt"                      
[162] "Nu Flavor"                     
[163] "Offspring, The"                
[164] "Paisley, Brad"                 
[165] "Papa Roach"                    
[166] "Pearl Jam"                     
[167] "Pink"                          
[168] "Price, Kelly"                  
[169] "Profyle"                       
[170] "Puff Daddy"                    
[171] "Q-Tip"                         
[172] "R.E.M."                        
[173] "Rascal Flatts"                 
[174] "Raye, Collin"                  
[175] "Red Hot Chili Peppers"         
[176] "Rimes, LeAnn"                  
[177] "Rogers, Kenny"                 
[178] "Ruff Endz"                     
[179] "Sammie"                        
[180] "Santana"                       
[181] "Savage Garden"                 
[182] "SheDaisy"                      
[183] "Sheist, Shade"                 
[184] "Shyne"                         
[185] "Simpson, Jessica"              
[186] "Sisqo"                         
[187] "Sister Hazel"                  
[188] "Smash Mouth"                   
[189] "Smith, Will"                   
[190] "Son By Four"                   
[191] "Sonique"                       
[192] "SoulDecision"                  
[193] "Spears, Britney"               
[194] "Spencer, Tracie"               
[195] "Splender"                      
[196] "Sting"                         
[197] "Stone Temple Pilots"           
[198] "Stone, Angie"                  
[199] "Strait, George"                
[200] "Sugar Ray"                     
[201] "TLC"                           
[202] "Tamar"                         
[203] "Tamia"                         
[204] "Third Eye Blind"               
[205] "Thomas, Carl"                  
[206] "Tippin, Aaron"                 
[207] "Train"                         
[208] "Trick Daddy"                   
[209] "Trina"                         
[210] "Tritt, Travis"                 
[211] "Tuesday"                       
[212] "Urban, Keith"                  
[213] "Usher"                         
[214] "Vassar, Phil"                  
[215] "Vertical Horizon"              
[216] "Vitamin C"                     
[217] "Walker, Clay"                  
[218] "Wallflowers, The"              
[219] "Westlife"                      
[220] "Williams, Robbie"              
[221] "Wills, Mark"                   
[222] "Worley, Darryl"                
[223] "Wright, Chely"                 
[224] "Yankee Grey"                   
[225] "Yearwood, Trisha"              
[226] "Ying Yang Twins"               
[227] "Zombie Nation"                 
[228] "matchbox twenty"               
\end{verbatim}
\end{itemize}

\subsection{\texorpdfstring{Filtrando linhas: \texttt{filter},
\texttt{slice}}{Filtrando linhas: filter, slice}}\label{filtrando-linhas-filter-slice}

\begin{itemize}
\item
  Apenas as músicas da Britney Spears:

\begin{Shaded}
\begin{Highlighting}[]
\NormalTok{bb\_tidy }\SpecialCharTok{\%\textgreater{}\%} 
  \FunctionTok{filter}\NormalTok{(artista }\SpecialCharTok{==} \StringTok{\textquotesingle{}Spears, Britney\textquotesingle{}}\NormalTok{)}
\end{Highlighting}
\end{Shaded}

\begin{verbatim}
# A tibble: 51 x 5
  artista         musica                  entrou     semana   pos
  <chr>           <chr>                   <date>      <int> <dbl>
1 Spears, Britney From The Bottom Of M... 2000-01-29      1    76
2 Spears, Britney From The Bottom Of M... 2000-01-29      2    59
3 Spears, Britney From The Bottom Of M... 2000-01-29      3    52
4 Spears, Britney From The Bottom Of M... 2000-01-29      4    52
5 Spears, Britney From The Bottom Of M... 2000-01-29      5    14
6 Spears, Britney From The Bottom Of M... 2000-01-29      6    14
# i 45 more rows
\end{verbatim}
\item
  Apenas músicas que chegaram à posição $1$, sem mostrar a coluna
  \texttt{pos}:

\begin{Shaded}
\begin{Highlighting}[]
\NormalTok{bb\_tidy }\SpecialCharTok{\%\textgreater{}\%} 
  \FunctionTok{filter}\NormalTok{(pos }\SpecialCharTok{==} \DecValTok{1}\NormalTok{) }\SpecialCharTok{\%\textgreater{}\%} 
  \FunctionTok{select}\NormalTok{(}\SpecialCharTok{{-}}\NormalTok{pos)}
\end{Highlighting}
\end{Shaded}

\begin{verbatim}
# A tibble: 55 x 4
  artista             musica                  entrou     semana
  <chr>               <chr>                   <date>      <int>
1 Aaliyah             Try Again               2000-03-18     14
2 Aguilera, Christina Come On Over Baby (A... 2000-08-05     11
3 Aguilera, Christina Come On Over Baby (A... 2000-08-05     12
4 Aguilera, Christina Come On Over Baby (A... 2000-08-05     13
5 Aguilera, Christina Come On Over Baby (A... 2000-08-05     14
6 Aguilera, Christina What A Girl Wants       1999-11-27      8
# i 49 more rows
\end{verbatim}
\item
  Apenas músicas que chegaram à posição $1$ em menos de $10$ semanas,
  mostrando apenas artista e música:

\begin{Shaded}
\begin{Highlighting}[]
\NormalTok{bb\_tidy }\SpecialCharTok{\%\textgreater{}\%} 
  \FunctionTok{filter}\NormalTok{(pos }\SpecialCharTok{==} \DecValTok{1}\NormalTok{, semana }\SpecialCharTok{\textless{}} \DecValTok{10}\NormalTok{) }\SpecialCharTok{\%\textgreater{}\%} 
  \FunctionTok{distinct}\NormalTok{(artista, musica)}
\end{Highlighting}
\end{Shaded}

\begin{verbatim}
# A tibble: 5 x 2
  artista             musica                 
  <chr>               <chr>                  
1 Aguilera, Christina What A Girl Wants      
2 Destiny's Child     Independent Women Pa...
3 Madonna             Music                  
4 Santana             Maria, Maria           
5 Sisqo               Incomplete             
\end{verbatim}
\item
  As funções da família \texttt{slice} filtram linhas de diversas
  maneiras.
\item
  De acordo com seus índices (números de linha):

\begin{Shaded}
\begin{Highlighting}[]
\NormalTok{bb\_tidy }\SpecialCharTok{\%\textgreater{}\%} 
  \FunctionTok{slice}\NormalTok{(}\FunctionTok{c}\NormalTok{(}\DecValTok{1}\NormalTok{, }\DecValTok{1000}\NormalTok{, }\DecValTok{5000}\NormalTok{))}
\end{Highlighting}
\end{Shaded}

\begin{verbatim}
# A tibble: 3 x 5
  artista                 musica              entrou     semana   pos
  <chr>                   <chr>               <date>      <int> <dbl>
1 2 Pac                   Baby Don't Cry (Ke~ 2000-02-26      1    87
2 Clark Family Experience Meanwhile Back At ~ 2000-11-18      3    81
3 Vassar, Phil            Carlene             2000-03-04      3    64
\end{verbatim}

\begin{Shaded}
\begin{Highlighting}[]
\NormalTok{bb\_tidy }\SpecialCharTok{\%\textgreater{}\%} 
  \FunctionTok{slice\_head}\NormalTok{(}\AttributeTok{n =} \DecValTok{4}\NormalTok{)}
\end{Highlighting}
\end{Shaded}

\begin{verbatim}
# A tibble: 4 x 5
  artista musica                  entrou     semana   pos
  <chr>   <chr>                   <date>      <int> <dbl>
1 2 Pac   Baby Don't Cry (Keep... 2000-02-26      1    87
2 2 Pac   Baby Don't Cry (Keep... 2000-02-26      2    82
3 2 Pac   Baby Don't Cry (Keep... 2000-02-26      3    72
4 2 Pac   Baby Don't Cry (Keep... 2000-02-26      4    77
\end{verbatim}

\begin{Shaded}
\begin{Highlighting}[]
\NormalTok{bb\_tidy }\SpecialCharTok{\%\textgreater{}\%} 
  \FunctionTok{slice\_tail}\NormalTok{(}\AttributeTok{n =} \DecValTok{4}\NormalTok{)}
\end{Highlighting}
\end{Shaded}

\begin{verbatim}
# A tibble: 4 x 5
  artista         musica entrou     semana   pos
  <chr>           <chr>  <date>      <int> <dbl>
1 matchbox twenty Bent   2000-04-29     36    37
2 matchbox twenty Bent   2000-04-29     37    38
3 matchbox twenty Bent   2000-04-29     38    38
4 matchbox twenty Bent   2000-04-29     39    48
\end{verbatim}
\item
  De acordo com a {\hl{ordenação de uma coluna}} ou {\hl{de uma função
  das colunas}}:

\begin{Shaded}
\begin{Highlighting}[]
\NormalTok{bb\_tidy }\SpecialCharTok{\%\textgreater{}\%} 
  \FunctionTok{slice\_min}\NormalTok{(pos)}
\end{Highlighting}
\end{Shaded}

\begin{verbatim}
# A tibble: 55 x 5
  artista             musica                  entrou     semana   pos
  <chr>               <chr>                   <date>      <int> <dbl>
1 Aaliyah             Try Again               2000-03-18     14     1
2 Aguilera, Christina Come On Over Baby (A... 2000-08-05     11     1
3 Aguilera, Christina Come On Over Baby (A... 2000-08-05     12     1
4 Aguilera, Christina Come On Over Baby (A... 2000-08-05     13     1
5 Aguilera, Christina Come On Over Baby (A... 2000-08-05     14     1
6 Aguilera, Christina What A Girl Wants       1999-11-27      8     1
# i 49 more rows
\end{verbatim}

\begin{Shaded}
\begin{Highlighting}[]
\NormalTok{bb\_tidy }\SpecialCharTok{\%\textgreater{}\%} 
  \FunctionTok{slice\_max}\NormalTok{(semana)}
\end{Highlighting}
\end{Shaded}

\begin{verbatim}
# A tibble: 1 x 5
  artista musica entrou     semana   pos
  <chr>   <chr>  <date>      <int> <dbl>
1 Creed   Higher 1999-09-11     65    49
\end{verbatim}
\item
  Aleatoriamente, criando uma amostra:

\begin{Shaded}
\begin{Highlighting}[]
\NormalTok{bb\_tidy }\SpecialCharTok{\%\textgreater{}\%} 
  \FunctionTok{slice\_sample}\NormalTok{(}\AttributeTok{n =} \DecValTok{5}\NormalTok{)}
\end{Highlighting}
\end{Shaded}

\begin{verbatim}
# A tibble: 5 x 5
  artista         musica                  entrou     semana   pos
  <chr>           <chr>                   <date>      <int> <dbl>
1 Raye, Collin    Couldn't Last A Mome... 2000-03-18      4    73
2 Mystikal        Shake Ya Ass            2000-08-12      4    41
3 Savage Garden   I Knew I Loved You      1999-10-23     23     8
4 N'Sync          It's Gonna Be Me        2000-05-06      4    39
5 Lopez, Jennifer Feelin' Good            2000-02-19     18    82
\end{verbatim}
\item
  Veja a ajuda de \texttt{slice} para saber mais sobre estas funções.
  Por exemplo:

  \begin{itemize}
  \item
    \texttt{slice\_min} e \texttt{slice\_max} podem considerar ou não
    empates.
  \item
    Você pode especificar uma proporção de linhas (usando \texttt{prop})
    em vez da quantidade de linhas (\texttt{n}).
  \item
    Você pode fazer amostragem com reposição, ou com probabilidades
    diferentes para cada linha.
  \end{itemize}
\end{itemize}

\subsection{\texorpdfstring{Ordenando linhas:
\texttt{arrange}}{Ordenando linhas: arrange}}\label{ordenando-linhas-arrange}

\begin{itemize}
\item
  Por título, sem repetições:

\begin{Shaded}
\begin{Highlighting}[]
\NormalTok{bb\_tidy }\SpecialCharTok{\%\textgreater{}\%} 
  \FunctionTok{select}\NormalTok{(musica) }\SpecialCharTok{\%\textgreater{}\%} 
  \FunctionTok{distinct}\NormalTok{() }\SpecialCharTok{\%\textgreater{}\%} 
  \FunctionTok{arrange}\NormalTok{(musica)}
\end{Highlighting}
\end{Shaded}

\begin{verbatim}
# A tibble: 316 x 1
  musica                 
  <chr>                  
1 (Hot S**t) Country G...
2 3 Little Words         
3 911                    
4 A Country Boy Can Su...
5 A Little Gasoline      
6 A Puro Dolor (Purest...
# i 310 more rows
\end{verbatim}
\item
  Por título, sem repetições, em ordem inversa:

\begin{Shaded}
\begin{Highlighting}[]
\NormalTok{bb\_tidy }\SpecialCharTok{\%\textgreater{}\%} 
  \FunctionTok{select}\NormalTok{(musica) }\SpecialCharTok{\%\textgreater{}\%} 
  \FunctionTok{distinct}\NormalTok{() }\SpecialCharTok{\%\textgreater{}\%} 
  \FunctionTok{arrange}\NormalTok{(}\FunctionTok{desc}\NormalTok{(musica))}
\end{Highlighting}
\end{Shaded}

\begin{verbatim}
# A tibble: 316 x 1
  musica                 
  <chr>                  
1 www.memory             
2 Your Everything        
3 You're A God           
4 You'll Always Be Lov...
5 You Won't Be Lonely ...
6 You Should've Told M...
# i 310 more rows
\end{verbatim}
\end{itemize}

\subsection{\texorpdfstring{Contando linhas:
\texttt{count}}{Contando linhas: count}}\label{contando-linhas-count}

\begin{itemize}
\item
  Quantas semanas cada artista ficou nos \emph{top} $100$? Duas músicas
  na mesma semana contam como duas semanas.

\begin{Shaded}
\begin{Highlighting}[]
\NormalTok{bb\_tidy }\SpecialCharTok{\%\textgreater{}\%} 
  \FunctionTok{count}\NormalTok{(artista, }\AttributeTok{sort =} \ConstantTok{TRUE}\NormalTok{)}
\end{Highlighting}
\end{Shaded}

\begin{verbatim}
# A tibble: 228 x 2
  artista             n
  <chr>           <int>
1 Creed             104
2 Lonestar           95
3 Destiny's Child    92
4 N'Sync             74
5 Sisqo              74
6 3 Doors Down       73
# i 222 more rows
\end{verbatim}
\item
  Quantas semanas cada música ficou nos \emph{top} $100$?

\begin{Shaded}
\begin{Highlighting}[]
\NormalTok{bb\_tidy }\SpecialCharTok{\%\textgreater{}\%} 
  \FunctionTok{count}\NormalTok{(musica, }\AttributeTok{sort =} \ConstantTok{TRUE}\NormalTok{)}
\end{Highlighting}
\end{Shaded}

\begin{verbatim}
# A tibble: 316 x 2
  musica                  n
  <chr>               <int>
1 Higher                 57
2 Amazed                 55
3 Breathe                53
4 Kryptonite             53
5 With Arms Wide Open    47
6 I Wanna Know           44
# i 310 more rows
\end{verbatim}
\item
  Se houve músicas com o mesmo nome, mas de artistas diferentes, {\hl{o
  código acima está errado}}. O certo é

\begin{Shaded}
\begin{Highlighting}[]
\NormalTok{bb\_tidy }\SpecialCharTok{\%\textgreater{}\%} 
  \FunctionTok{count}\NormalTok{(musica, artista, }\AttributeTok{sort =} \ConstantTok{TRUE}\NormalTok{)}
\end{Highlighting}
\end{Shaded}

\begin{verbatim}
# A tibble: 317 x 3
  musica              artista          n
  <chr>               <chr>        <int>
1 Higher              Creed           57
2 Amazed              Lonestar        55
3 Breathe             Hill, Faith     53
4 Kryptonite          3 Doors Down    53
5 With Arms Wide Open Creed           47
6 I Wanna Know        Joe             44
# i 311 more rows
\end{verbatim}

  De fato, há uma diferença de uma linha.
\end{itemize}

\subsubsection{Exercício}\label{exercuxedcio-1}

\begin{itemize}
\item
  Ache o título da música que tem dois artistas diferentes.

  \textbf{Sugestão:} conte por música e artista primeiro, depois só por
  música.
\end{itemize}

\subsection{\texorpdfstring{Agrupando linhas: \texttt{group\_by} e
\texttt{summarize}}{Agrupando linhas: group\_by e summarize}}\label{agrupando-linhas-group_by-e-summarize}

\begin{itemize}
\item
  Qual foi a melhor posição que cada artista alcançou?

\begin{Shaded}
\begin{Highlighting}[]
\NormalTok{bb\_tidy }\SpecialCharTok{\%\textgreater{}\%} 
  \FunctionTok{group\_by}\NormalTok{(artista) }\SpecialCharTok{\%\textgreater{}\%} 
  \FunctionTok{summarize}\NormalTok{(}\AttributeTok{melhor =} \FunctionTok{min}\NormalTok{(pos)) }\SpecialCharTok{\%\textgreater{}\%} 
  \FunctionTok{arrange}\NormalTok{(melhor)}
\end{Highlighting}
\end{Shaded}

\begin{verbatim}
# A tibble: 228 x 2
  artista             melhor
  <chr>                <dbl>
1 Aaliyah                  1
2 Aguilera, Christina      1
3 Carey, Mariah            1
4 Creed                    1
5 Destiny's Child          1
6 Iglesias, Enrique        1
# i 222 more rows
\end{verbatim}
\item
  Qual foi a melhor posição que cada música alcançou?

\begin{Shaded}
\begin{Highlighting}[]
\NormalTok{bb\_tidy }\SpecialCharTok{\%\textgreater{}\%} 
  \FunctionTok{group\_by}\NormalTok{(artista, musica) }\SpecialCharTok{\%\textgreater{}\%} 
  \FunctionTok{summarize}\NormalTok{(}\AttributeTok{melhor =} \FunctionTok{min}\NormalTok{(pos)) }\SpecialCharTok{\%\textgreater{}\%} 
  \FunctionTok{arrange}\NormalTok{(melhor)}
\end{Highlighting}
\end{Shaded}

\begin{verbatim}
`summarise()` has grouped output by 'artista'. You can override
using the `.groups` argument.
\end{verbatim}

\begin{verbatim}
# A tibble: 317 x 3
# Groups:   artista [228]
  artista             musica                  melhor
  <chr>               <chr>                    <dbl>
1 Aaliyah             Try Again                    1
2 Aguilera, Christina Come On Over Baby (A...      1
3 Aguilera, Christina What A Girl Wants            1
4 Carey, Mariah       Thank God I Found Yo...      1
5 Creed               With Arms Wide Open          1
6 Destiny's Child     Independent Women Pa...      1
# i 311 more rows
\end{verbatim}
\item
  Quando usamos \texttt{summarize}, só o agrupamento {\hl{mais interno}}
  é desfeito. Isto significa que {\hl{o resultado acima ainda está
  agrupado por {\mbox{\texttt{artista}}}}}.
\item
  Quantas semanas cada artista ficou na posição $1$?

  A função \texttt{n()} é uma maneira conveniente de {\hl{obter o número
  de linhas de um grupo}} (ou, se não houver agrupamento, de toda a
  \emph{tibble}); mas {\hl{{\mbox{\texttt{n()}}} só pode ser chamada em
  certos contextos}}, como \texttt{summarise()} ou \texttt{mutate()}.

\begin{Shaded}
\begin{Highlighting}[]
\NormalTok{bb\_tidy }\SpecialCharTok{\%\textgreater{}\%} 
  \FunctionTok{filter}\NormalTok{(pos }\SpecialCharTok{==} \DecValTok{1}\NormalTok{) }\SpecialCharTok{\%\textgreater{}\%} 
  \FunctionTok{group\_by}\NormalTok{(artista) }\SpecialCharTok{\%\textgreater{}\%}
  \FunctionTok{summarize}\NormalTok{(}\AttributeTok{semanas =} \FunctionTok{n}\NormalTok{()) }\SpecialCharTok{\%\textgreater{}\%} 
  \FunctionTok{arrange}\NormalTok{(}\FunctionTok{desc}\NormalTok{(semanas))}
\end{Highlighting}
\end{Shaded}

\begin{verbatim}
# A tibble: 15 x 2
  artista             semanas
  <chr>                 <int>
1 Destiny's Child          14
2 Santana                  10
3 Aguilera, Christina       6
4 Madonna                   4
5 Savage Garden             4
6 Iglesias, Enrique         3
# i 9 more rows
\end{verbatim}
\item
  Perceba que \texttt{count}, que vimos mais acima, faz agrupamentos do
  mesmo modo:

\begin{Shaded}
\begin{Highlighting}[]
\NormalTok{bb\_tidy }\SpecialCharTok{\%\textgreater{}\%} 
  \FunctionTok{filter}\NormalTok{(pos }\SpecialCharTok{==} \DecValTok{1}\NormalTok{) }\SpecialCharTok{\%\textgreater{}\%} 
  \FunctionTok{count}\NormalTok{(artista, }\AttributeTok{sort =} \ConstantTok{TRUE}\NormalTok{)}
\end{Highlighting}
\end{Shaded}

\begin{verbatim}
# A tibble: 15 x 2
  artista                 n
  <chr>               <int>
1 Destiny's Child        14
2 Santana                10
3 Aguilera, Christina     6
4 Madonna                 4
5 Savage Garden           4
6 Iglesias, Enrique       3
# i 9 more rows
\end{verbatim}
\item
  Uma pergunta diferente: quais são os artistas cujas músicas apareceram
  no \emph{top} $100$ mais tempo depois do lançamento da música?

\begin{Shaded}
\begin{Highlighting}[]
\NormalTok{bb\_tidy }\SpecialCharTok{\%\textgreater{}\%} 
  \FunctionTok{group\_by}\NormalTok{(artista) }\SpecialCharTok{\%\textgreater{}\%} 
  \FunctionTok{summarize}\NormalTok{(}\AttributeTok{semanas =} \FunctionTok{max}\NormalTok{(semana)) }\SpecialCharTok{\%\textgreater{}\%} 
  \FunctionTok{arrange}\NormalTok{(}\FunctionTok{desc}\NormalTok{(semanas))}
\end{Highlighting}
\end{Shaded}

\begin{verbatim}
# A tibble: 228 x 2
  artista          semanas
  <chr>              <int>
1 Creed                 65
2 Lonestar              64
3 3 Doors Down          53
4 Hill, Faith           53
5 Joe                   44
6 Vertical Horizon      41
# i 222 more rows
\end{verbatim}
\item
  Qual a posição média de cada música? Lembre-se de que eliminamos as
  linhas com \texttt{NA}; logo, {\hl{a média vai ser sobre a quantidade
  de semanas em que a música esteve na lista}}.

  ::: \{.cell layout-align=``center''\}

\begin{Shaded}
\begin{Highlighting}[]
\NormalTok{media1 }\OtherTok{\textless{}{-}}\NormalTok{ bb\_tidy }\SpecialCharTok{\%\textgreater{}\%} 
  \FunctionTok{group\_by}\NormalTok{(artista, musica) }\SpecialCharTok{\%\textgreater{}\%} 
  \FunctionTok{summarize}\NormalTok{(}\AttributeTok{media =} \FunctionTok{mean}\NormalTok{(pos), }\AttributeTok{.groups =} \StringTok{\textquotesingle{}drop\textquotesingle{}}\NormalTok{) }\SpecialCharTok{\%\textgreater{}\%} 
  \FunctionTok{arrange}\NormalTok{(media)}
\end{Highlighting}
\end{Shaded}

  :::

\begin{Shaded}
\begin{Highlighting}[]
\NormalTok{media1}
\end{Highlighting}
\end{Shaded}

\begin{verbatim}
# A tibble: 317 x 3
  artista                          musica                  media
  <chr>                            <chr>                   <dbl>
1 "Santana"                        Maria, Maria             10.5
2 "Madonna"                        Music                    13.5
3 "N'Sync"                         Bye Bye Bye              14.3
4 "Elliott, Missy \"Misdemeanor\"" Hot Boyz                 14.3
5 "Destiny's Child"                Independent Women Pa...  14.8
6 "Iglesias, Enrique"              Be With You              15.8
# i 311 more rows
\end{verbatim}
\item
  E se quisermos {\hl{a média sobre o número de semanas desde a entrada
  da música até a última semana}} em que a música apareceu na lista?

  ::: \{.cell layout-align=``center''\}

\begin{Shaded}
\begin{Highlighting}[]
\NormalTok{media2 }\OtherTok{\textless{}{-}}\NormalTok{ bb\_tidy }\SpecialCharTok{\%\textgreater{}\%} 
  \FunctionTok{group\_by}\NormalTok{(artista, musica) }\SpecialCharTok{\%\textgreater{}\%} 
  \FunctionTok{summarize}\NormalTok{(}\AttributeTok{media =} \FunctionTok{sum}\NormalTok{(pos)}\SpecialCharTok{/}\FunctionTok{max}\NormalTok{(semana), }\AttributeTok{.groups =} \StringTok{\textquotesingle{}drop\textquotesingle{}}\NormalTok{) }\SpecialCharTok{\%\textgreater{}\%} 
  \FunctionTok{arrange}\NormalTok{(media)}
\end{Highlighting}
\end{Shaded}

  :::

\begin{Shaded}
\begin{Highlighting}[]
\NormalTok{media2}
\end{Highlighting}
\end{Shaded}

\begin{verbatim}
# A tibble: 317 x 3
  artista                          musica                  media
  <chr>                            <chr>                   <dbl>
1 "Santana"                        Maria, Maria             10.5
2 "Madonna"                        Music                    13.5
3 "N'Sync"                         Bye Bye Bye              14.3
4 "Elliott, Missy \"Misdemeanor\"" Hot Boyz                 14.3
5 "Destiny's Child"                Independent Women Pa...  14.8
6 "Iglesias, Enrique"              Be With You              15.8
# i 311 more rows
\end{verbatim}

  As primeiras linhas são iguais, mas os resultados são diferentes:

  ::: \{.cell layout-align=``center''\}

\begin{Shaded}
\begin{Highlighting}[]
\FunctionTok{identical}\NormalTok{(media1, media2)}
\end{Highlighting}
\end{Shaded}

  ::: \{.cell-output .cell-output-stdout\}

\begin{verbatim}
[1] FALSE
\end{verbatim}

  ::: :::
\end{itemize}

\section{Exercícios}\label{exercuxedcios-1}

\begin{enumerate}
\def\labelenumi{\arabic{enumi}.}
\item
  Vamos trabalhar com um conjunto de dados sobre super-heróis.

  Carregue o \texttt{tidyverse} com o comando

  ::: \{.cell layout-align=``center''\}

\begin{Shaded}
\begin{Highlighting}[]
\FunctionTok{library}\NormalTok{(tidyverse)}
\end{Highlighting}
\end{Shaded}

  :::

  Execute o seguinte comando para ler os dados para uma \emph{tibble}:

  ::: \{.cell layout-align=``center''\}

\begin{Shaded}
\begin{Highlighting}[]
\NormalTok{arquivo }\OtherTok{\textless{}{-}} \FunctionTok{paste0}\NormalTok{(}
  \StringTok{\textquotesingle{}https://github.com/fnaufel/\textquotesingle{}}\NormalTok{,}
  \StringTok{\textquotesingle{}probestr/raw/master/data/\textquotesingle{}}\NormalTok{,}
  \StringTok{\textquotesingle{}heroes\_information.csv\textquotesingle{}}      
\NormalTok{)}

\NormalTok{herois\_info }\OtherTok{\textless{}{-}} \FunctionTok{read\_csv}\NormalTok{(}
\NormalTok{  arquivo,}
  \AttributeTok{na =} \FunctionTok{c}\NormalTok{(}\StringTok{\textquotesingle{}\textquotesingle{}}\NormalTok{, }\StringTok{\textquotesingle{}{-}\textquotesingle{}}\NormalTok{, }\StringTok{\textquotesingle{}NA\textquotesingle{}}\NormalTok{)}
\NormalTok{) }\SpecialCharTok{\%\textgreater{}\%} 
  \CommentTok{\# Eliminar a primeira coluna (números de série)}
  \FunctionTok{select}\NormalTok{(}\SpecialCharTok{{-}}\DecValTok{1}\NormalTok{) }\SpecialCharTok{\%\textgreater{}\%} 
  \CommentTok{\# Renomear colunas restantes}
  \FunctionTok{rename}\NormalTok{(}
    \AttributeTok{nome =}\NormalTok{ name,}
    \AttributeTok{sexo =}\NormalTok{ Gender,}
    \AttributeTok{olhos =} \StringTok{\textquotesingle{}Eye color\textquotesingle{}}\NormalTok{,}
\NormalTok{    raça }\OtherTok{=}\NormalTok{ Race,}
    \AttributeTok{cabelos =} \StringTok{\textquotesingle{}Hair color\textquotesingle{}}\NormalTok{,}
    \AttributeTok{altura =}\NormalTok{ Height,}
    \AttributeTok{editora =}\NormalTok{ Publisher,}
    \AttributeTok{pele =} \StringTok{\textquotesingle{}Skin color\textquotesingle{}}\NormalTok{,}
    \AttributeTok{lado =}\NormalTok{ Alignment,}
    \AttributeTok{peso =}\NormalTok{ Weight}
\NormalTok{  )}
\end{Highlighting}
\end{Shaded}

  :::
\item
  Quantas linhas tem a \emph{tibble}?

  \begin{tcolorbox}[enhanced jigsaw, coltitle=black, colbacktitle=quarto-callout-tip-color!10!white, title=\textcolor{quarto-callout-tip-color}{\faLightbulb}\hspace{0.5em}{Resposta}, toprule=.15mm, leftrule=.75mm, opacityback=0, colback=white, arc=.35mm, breakable, bottomtitle=1mm, left=2mm, toptitle=1mm, titlerule=0mm, rightrule=.15mm, bottomrule=.15mm, opacitybacktitle=0.6, colframe=quarto-callout-tip-color-frame]

\begin{Shaded}
\begin{Highlighting}[]
\NormalTok{herois\_info }\SpecialCharTok{\%\textgreater{}\%} \FunctionTok{nrow}\NormalTok{()}
\end{Highlighting}
\end{Shaded}

\begin{verbatim}
[1] 734
\end{verbatim}

  \end{tcolorbox}
\item
  Existem heróis que aparecem em mais de uma linha?

  \begin{tcolorbox}[enhanced jigsaw, coltitle=black, colbacktitle=quarto-callout-tip-color!10!white, title=\textcolor{quarto-callout-tip-color}{\faLightbulb}\hspace{0.5em}{Resposta}, toprule=.15mm, leftrule=.75mm, opacityback=0, colback=white, arc=.35mm, breakable, bottomtitle=1mm, left=2mm, toptitle=1mm, titlerule=0mm, rightrule=.15mm, bottomrule=.15mm, opacitybacktitle=0.6, colframe=quarto-callout-tip-color-frame]

\begin{Shaded}
\begin{Highlighting}[]
\NormalTok{herois\_info }\SpecialCharTok{\%\textgreater{}\%} 
  \FunctionTok{count}\NormalTok{(nome)}
\end{Highlighting}
\end{Shaded}

\begin{verbatim}
# A tibble: 715 x 2
  nome              n
  <chr>         <int>
1 A-Bomb            1
2 Abe Sapien        1
3 Abin Sur          1
4 Abomination       1
5 Abraxas           1
6 Absorbing Man     1
# i 709 more rows
\end{verbatim}

  Precisaríamos {\hl{examinar a tabela acima}}, procurando linhas com
  $n > 1$.

  {\hl{Vamos pedir para o R fazer isto:}}

\begin{Shaded}
\begin{Highlighting}[]
\NormalTok{repetidos }\OtherTok{\textless{}{-}}\NormalTok{ herois\_info }\SpecialCharTok{\%\textgreater{}\%} 
  \FunctionTok{count}\NormalTok{(nome) }\SpecialCharTok{\%\textgreater{}\%} 
  \FunctionTok{filter}\NormalTok{(n }\SpecialCharTok{\textgreater{}} \DecValTok{1}\NormalTok{)}

\NormalTok{repetidos}
\end{Highlighting}
\end{Shaded}

\begin{verbatim}
# A tibble: 17 x 2
  nome             n
  <chr>        <int>
1 Angel            2
2 Atlas            2
3 Atom             2
4 Batgirl          2
5 Batman           2
6 Black Canary     2
# i 11 more rows
\end{verbatim}

  Vamos mostrar mais dados destes heróis:

\begin{Shaded}
\begin{Highlighting}[]
\NormalTok{herois\_info }\SpecialCharTok{\%\textgreater{}\%} 
  \FunctionTok{filter}\NormalTok{(nome }\SpecialCharTok{\%in\%}\NormalTok{ repetidos}\SpecialCharTok{$}\NormalTok{nome) }\SpecialCharTok{\%\textgreater{}\%} 
  \FunctionTok{select}\NormalTok{(nome, editora, raça, }\FunctionTok{everything}\NormalTok{())}
\end{Highlighting}
\end{Shaded}

\begin{verbatim}
# A tibble: 36 x 10
  nome  editora    raça  sexo  olhos cabelos altura pele  lado   peso
  <chr> <chr>      <chr> <chr> <chr> <chr>    <dbl> <chr> <chr> <dbl>
1 Angel Marvel Co~ <NA>  Male  blue  Blond      183 <NA>  good     68
2 Angel Dark Hors~ Vamp~ Male  <NA>  <NA>       -99 <NA>  good    -99
3 Atlas Marvel Co~ Muta~ Male  brown Red        183 <NA>  good    101
4 Atlas DC Comics  God ~ Male  blue  Brown      198 <NA>  bad     126
5 Atom  DC Comics  <NA>  Male  blue  Red        178 <NA>  good     68
6 Atom  DC Comics  <NA>  Male  <NA>  <NA>       -99 <NA>  good    -99
# i 30 more rows
\end{verbatim}

  Em alguns casos, são editoras diferentes (como para Angel e Atlas).

  Em alguns casos, o mesmo herói aparece com várias características.

  São $17$ heróis que aparecem mais de uma vez. É um número pequeno o
  bastante para corrigirmos a situação manualmente.

  Como não tenho conhecimento suficiente sobre heróis para fazer isso,
  vou ignorar esta confusão e usar os dados como estão.

  \end{tcolorbox}
\item
  Quantas editoras diferentes existem na \emph{tibble}? Liste-as em
  ordem decrescente de quantidade de heróis.

  \begin{tcolorbox}[enhanced jigsaw, coltitle=black, colbacktitle=quarto-callout-tip-color!10!white, title=\textcolor{quarto-callout-tip-color}{\faLightbulb}\hspace{0.5em}{Resposta}, toprule=.15mm, leftrule=.75mm, opacityback=0, colback=white, arc=.35mm, breakable, bottomtitle=1mm, left=2mm, toptitle=1mm, titlerule=0mm, rightrule=.15mm, bottomrule=.15mm, opacitybacktitle=0.6, colframe=quarto-callout-tip-color-frame]

\begin{Shaded}
\begin{Highlighting}[]
\NormalTok{herois\_info }\SpecialCharTok{\%\textgreater{}\%} \FunctionTok{count}\NormalTok{(editora, }\AttributeTok{sort =} \ConstantTok{TRUE}\NormalTok{)}
\end{Highlighting}
\end{Shaded}

\begin{verbatim}
# A tibble: 25 x 2
  editora               n
  <chr>             <int>
1 Marvel Comics       388
2 DC Comics           215
3 NBC - Heroes         19
4 Dark Horse Comics    18
5 <NA>                 15
6 George Lucas         14
# i 19 more rows
\end{verbatim}

  \end{tcolorbox}
\item
  Vamos colocar todas as editores menores em uma classe só.

  Na coluna \texttt{editora}, substitua

  \begin{itemize}
  \tightlist
  \item
    `Marvel Comics' por `Marvel',
  \item
    `DC Comics' por `DC', e
  \item
    todas as outras editoras pelo termo `Outras'.
  \end{itemize}

  \textbf{Dica:} use a função \texttt{case\_when()}, do
  \texttt{tidyverse}.

  \begin{tcolorbox}[enhanced jigsaw, coltitle=black, colbacktitle=quarto-callout-tip-color!10!white, title=\textcolor{quarto-callout-tip-color}{\faLightbulb}\hspace{0.5em}{Resposta}, toprule=.15mm, leftrule=.75mm, opacityback=0, colback=white, arc=.35mm, breakable, bottomtitle=1mm, left=2mm, toptitle=1mm, titlerule=0mm, rightrule=.15mm, bottomrule=.15mm, opacitybacktitle=0.6, colframe=quarto-callout-tip-color-frame]

\begin{Shaded}
\begin{Highlighting}[]
\NormalTok{herois\_info }\OtherTok{\textless{}{-}}\NormalTok{ herois\_info }\SpecialCharTok{\%\textgreater{}\%} 
  \FunctionTok{mutate}\NormalTok{(}
    \AttributeTok{editora =} \FunctionTok{case\_when}\NormalTok{(}
\NormalTok{      editora }\SpecialCharTok{==} \StringTok{\textquotesingle{}Marvel Comics\textquotesingle{}} \SpecialCharTok{\textasciitilde{}} \StringTok{\textquotesingle{}Marvel\textquotesingle{}}\NormalTok{,}
\NormalTok{      editora }\SpecialCharTok{==} \StringTok{\textquotesingle{}DC Comics\textquotesingle{}} \SpecialCharTok{\textasciitilde{}} \StringTok{\textquotesingle{}DC\textquotesingle{}}\NormalTok{,}
      \ConstantTok{TRUE} \SpecialCharTok{\textasciitilde{}} \StringTok{\textquotesingle{}Outras\textquotesingle{}}
\NormalTok{    )}
\NormalTok{)}
\end{Highlighting}
\end{Shaded}

  \end{tcolorbox}
\item
  Confira, novamente, a quantidade de valores diferentes na coluna
  \texttt{editora}.

  \begin{tcolorbox}[enhanced jigsaw, coltitle=black, colbacktitle=quarto-callout-tip-color!10!white, title=\textcolor{quarto-callout-tip-color}{\faLightbulb}\hspace{0.5em}{Resposta}, toprule=.15mm, leftrule=.75mm, opacityback=0, colback=white, arc=.35mm, breakable, bottomtitle=1mm, left=2mm, toptitle=1mm, titlerule=0mm, rightrule=.15mm, bottomrule=.15mm, opacitybacktitle=0.6, colframe=quarto-callout-tip-color-frame]

\begin{Shaded}
\begin{Highlighting}[]
\NormalTok{herois\_info }\SpecialCharTok{\%\textgreater{}\%} \FunctionTok{count}\NormalTok{(editora, }\AttributeTok{sort =} \ConstantTok{TRUE}\NormalTok{)}
\end{Highlighting}
\end{Shaded}

\begin{verbatim}
# A tibble: 3 x 2
  editora     n
  <chr>   <int>
1 Marvel    388
2 DC        215
3 Outras    131
\end{verbatim}

  \end{tcolorbox}
\item
  Existem heróis sem informação de editora. Quantos? Quais são?

  \begin{tcolorbox}[enhanced jigsaw, coltitle=black, colbacktitle=quarto-callout-tip-color!10!white, title=\textcolor{quarto-callout-tip-color}{\faLightbulb}\hspace{0.5em}{Resposta}, toprule=.15mm, leftrule=.75mm, opacityback=0, colback=white, arc=.35mm, breakable, bottomtitle=1mm, left=2mm, toptitle=1mm, titlerule=0mm, rightrule=.15mm, bottomrule=.15mm, opacitybacktitle=0.6, colframe=quarto-callout-tip-color-frame]

\begin{Shaded}
\begin{Highlighting}[]
\NormalTok{herois\_info }\SpecialCharTok{\%\textgreater{}\%} \FunctionTok{filter}\NormalTok{(}\FunctionTok{is.na}\NormalTok{(editora))}
\end{Highlighting}
\end{Shaded}

\begin{verbatim}
# A tibble: 0 x 10
# i 10 variables: nome <chr>, sexo <chr>, olhos <chr>, raça <chr>,
#   cabelos <chr>, altura <dbl>, editora <chr>, pele <chr>,
#   lado <chr>, peso <dbl>
\end{verbatim}

  Na verdade, a chamada a \texttt{case\_when()}, da maneira como fiz, já
  substituiu os \texttt{NA} por `Outras'. Entenda por quê.

  \end{tcolorbox}
\item
  Altere novamente a coluna \texttt{editora}, colocando o valor `Outras'
  para os heróis sem informação de editora. Use a função
  \texttt{if\_else()} (com \emph{underscore}, não a função
  \texttt{ifelse}).

  \begin{tcolorbox}[enhanced jigsaw, coltitle=black, colbacktitle=quarto-callout-tip-color!10!white, title=\textcolor{quarto-callout-tip-color}{\faLightbulb}\hspace{0.5em}{Resposta}, toprule=.15mm, leftrule=.75mm, opacityback=0, colback=white, arc=.35mm, breakable, bottomtitle=1mm, left=2mm, toptitle=1mm, titlerule=0mm, rightrule=.15mm, bottomrule=.15mm, opacitybacktitle=0.6, colframe=quarto-callout-tip-color-frame]

  Se, no seu caso, ainda houver valores \texttt{NA} em \texttt{editora},
  basta fazer o seguinte:

\begin{Shaded}
\begin{Highlighting}[]
\NormalTok{herois\_info }\OtherTok{\textless{}{-}}\NormalTok{ herois\_info }\SpecialCharTok{\%\textgreater{}\%} 
  \FunctionTok{mutate}\NormalTok{(}
    \AttributeTok{editora =} \FunctionTok{if\_else}\NormalTok{(}\FunctionTok{is.na}\NormalTok{(editora), }\StringTok{\textquotesingle{}Outras\textquotesingle{}}\NormalTok{, editora)}
\NormalTok{  )}
\end{Highlighting}
\end{Shaded}

  \end{tcolorbox}
\item
  Confira, mais uma vez, a quantidade de valores diferentes na coluna
  \texttt{editora}.

  \begin{tcolorbox}[enhanced jigsaw, coltitle=black, colbacktitle=quarto-callout-tip-color!10!white, title=\textcolor{quarto-callout-tip-color}{\faLightbulb}\hspace{0.5em}{Resposta}, toprule=.15mm, leftrule=.75mm, opacityback=0, colback=white, arc=.35mm, breakable, bottomtitle=1mm, left=2mm, toptitle=1mm, titlerule=0mm, rightrule=.15mm, bottomrule=.15mm, opacitybacktitle=0.6, colframe=quarto-callout-tip-color-frame]

\begin{Shaded}
\begin{Highlighting}[]
\NormalTok{herois\_info }\SpecialCharTok{\%\textgreater{}\%} \FunctionTok{count}\NormalTok{(editora, }\AttributeTok{sort =} \ConstantTok{TRUE}\NormalTok{)}
\end{Highlighting}
\end{Shaded}

\begin{verbatim}
# A tibble: 3 x 2
  editora     n
  <chr>   <int>
1 Marvel    388
2 DC        215
3 Outras    131
\end{verbatim}

  \end{tcolorbox}
\item
  Existem heróis sem informação de sexo? Quantos? Para estes heróis,
  coloque o valor `Desconhecido' na coluna \texttt{sexo}.

  \begin{tcolorbox}[enhanced jigsaw, coltitle=black, colbacktitle=quarto-callout-tip-color!10!white, title=\textcolor{quarto-callout-tip-color}{\faLightbulb}\hspace{0.5em}{Resposta}, toprule=.15mm, leftrule=.75mm, opacityback=0, colback=white, arc=.35mm, breakable, bottomtitle=1mm, left=2mm, toptitle=1mm, titlerule=0mm, rightrule=.15mm, bottomrule=.15mm, opacitybacktitle=0.6, colframe=quarto-callout-tip-color-frame]

\begin{Shaded}
\begin{Highlighting}[]
\NormalTok{herois\_info }\SpecialCharTok{\%\textgreater{}\%} \FunctionTok{filter}\NormalTok{(}\FunctionTok{is.na}\NormalTok{(sexo))}
\end{Highlighting}
\end{Shaded}

\begin{verbatim}
# A tibble: 29 x 10
  nome     sexo  olhos raça  cabelos altura editora pele  lado   peso
  <chr>    <chr> <chr> <chr> <chr>    <dbl> <chr>   <chr> <chr> <dbl>
1 Bird-Br~ <NA>  <NA>  <NA>  <NA>       -99 Marvel  <NA>  good    -99
2 Blaques~ <NA>  black <NA>  No Hair    -99 Marvel  <NA>  good    -99
3 Box III  <NA>  blue  <NA>  Blond      193 Marvel  <NA>  good    110
4 Box IV   <NA>  brown <NA>  Brown ~    -99 Marvel  <NA>  good    -99
5 Captain~ <NA>  <NA>  God ~ <NA>       -99 Marvel  <NA>  good    -99
6 Cecilia~ <NA>  brown <NA>  Brown      170 Marvel  <NA>  good     62
# i 23 more rows
\end{verbatim}

\begin{Shaded}
\begin{Highlighting}[]
\NormalTok{herois\_info }\OtherTok{\textless{}{-}}\NormalTok{  herois\_info }\SpecialCharTok{\%\textgreater{}\%} 
  \FunctionTok{mutate}\NormalTok{(}
    \AttributeTok{sexo =} \FunctionTok{if\_else}\NormalTok{(}
      \FunctionTok{is.na}\NormalTok{(sexo),}
      \StringTok{\textquotesingle{}Desconhecido\textquotesingle{}}\NormalTok{,}
\NormalTok{      sexo}
\NormalTok{    )}
\NormalTok{  )}
\end{Highlighting}
\end{Shaded}

  Conferindo:

\begin{Shaded}
\begin{Highlighting}[]
\NormalTok{herois\_info }\SpecialCharTok{\%\textgreater{}\%} \FunctionTok{filter}\NormalTok{(}\FunctionTok{is.na}\NormalTok{(sexo))}
\end{Highlighting}
\end{Shaded}

\begin{verbatim}
# A tibble: 0 x 10
# i 10 variables: nome <chr>, sexo <chr>, olhos <chr>, raça <chr>,
#   cabelos <chr>, altura <dbl>, editora <chr>, pele <chr>,
#   lado <chr>, peso <dbl>
\end{verbatim}

  \end{tcolorbox}
\item
  Qual a altura mínima? Qual a altura máxima? Substitua as alturas
  negativas por \texttt{NA}.

  \begin{tcolorbox}[enhanced jigsaw, coltitle=black, colbacktitle=quarto-callout-tip-color!10!white, title=\textcolor{quarto-callout-tip-color}{\faLightbulb}\hspace{0.5em}{Resposta}, toprule=.15mm, leftrule=.75mm, opacityback=0, colback=white, arc=.35mm, breakable, bottomtitle=1mm, left=2mm, toptitle=1mm, titlerule=0mm, rightrule=.15mm, bottomrule=.15mm, opacitybacktitle=0.6, colframe=quarto-callout-tip-color-frame]

  Podemos extrair o vetor de alturas com \texttt{pull} e usar a função
  \texttt{summary} do R base, que retorna um vetor:

\begin{Shaded}
\begin{Highlighting}[]
\NormalTok{herois\_info }\SpecialCharTok{\%\textgreater{}\%} 
  \FunctionTok{pull}\NormalTok{(altura) }\SpecialCharTok{\%\textgreater{}\%} 
  \FunctionTok{summary}\NormalTok{()}
\end{Highlighting}
\end{Shaded}

\begin{verbatim}
   Min. 1st Qu.  Median    Mean 3rd Qu.    Max. 
  -99,0   -99,0   175,0   102,3   185,0   975,0 
\end{verbatim}

  Ou podemos usar \texttt{summarize}, do \texttt{tidyverse}, que retorna
  uma \emph{tibble}:

\begin{Shaded}
\begin{Highlighting}[]
\NormalTok{herois\_info }\SpecialCharTok{\%\textgreater{}\%} 
  \FunctionTok{summarize}\NormalTok{(}
    \AttributeTok{minimo =} \FunctionTok{min}\NormalTok{(altura),}
    \AttributeTok{maximo =} \FunctionTok{max}\NormalTok{(altura)}
\NormalTok{  )}
\end{Highlighting}
\end{Shaded}

\begin{verbatim}
# A tibble: 1 x 2
  minimo maximo
   <dbl>  <dbl>
1    -99    975
\end{verbatim}

  Quantas alturas negativas existem?

\begin{Shaded}
\begin{Highlighting}[]
\NormalTok{herois\_info }\SpecialCharTok{\%\textgreater{}\%} \FunctionTok{count}\NormalTok{(altura }\SpecialCharTok{\textless{}} \DecValTok{0}\NormalTok{)}
\end{Highlighting}
\end{Shaded}

\begin{verbatim}
# A tibble: 2 x 2
  `altura < 0`     n
  <lgl>        <int>
1 FALSE          517
2 TRUE           217
\end{verbatim}

  Substituindo as alturas negativas por \texttt{NA}:

\begin{Shaded}
\begin{Highlighting}[]
\NormalTok{herois\_info }\OtherTok{\textless{}{-}}\NormalTok{ herois\_info }\SpecialCharTok{\%\textgreater{}\%} 
  \FunctionTok{mutate}\NormalTok{(}
    \AttributeTok{altura =} \FunctionTok{if\_else}\NormalTok{(}
\NormalTok{      altura }\SpecialCharTok{\textless{}} \DecValTok{0}\NormalTok{,}
      \ConstantTok{NA\_real\_}\NormalTok{,}
\NormalTok{      altura}
\NormalTok{    )}
\NormalTok{  )}
\end{Highlighting}
\end{Shaded}

  Como \texttt{if\_else} (com \emph{underscore}) é exigente, precisamos
  passar \texttt{NA\_real\_} em vez de \texttt{NA}.

  Situação atual:

\begin{Shaded}
\begin{Highlighting}[]
\NormalTok{herois\_info }\SpecialCharTok{\%\textgreater{}\%} 
  \FunctionTok{pull}\NormalTok{(altura) }\SpecialCharTok{\%\textgreater{}\%} 
  \FunctionTok{summary}\NormalTok{()}
\end{Highlighting}
\end{Shaded}

\begin{verbatim}
   Min. 1st Qu.  Median    Mean 3rd Qu.    Max.    NA's 
   15,2   173,0   183,0   186,7   191,0   975,0     217 
\end{verbatim}

  \end{tcolorbox}
\item
  Qual o peso mínimo? Qual o peso máximo? Substitua os pesos negativos
  por \texttt{NA}.

  \begin{tcolorbox}[enhanced jigsaw, coltitle=black, colbacktitle=quarto-callout-tip-color!10!white, title=\textcolor{quarto-callout-tip-color}{\faLightbulb}\hspace{0.5em}{Resposta}, toprule=.15mm, leftrule=.75mm, opacityback=0, colback=white, arc=.35mm, breakable, bottomtitle=1mm, left=2mm, toptitle=1mm, titlerule=0mm, rightrule=.15mm, bottomrule=.15mm, opacitybacktitle=0.6, colframe=quarto-callout-tip-color-frame]

  Como fizemos com as alturas:

\begin{Shaded}
\begin{Highlighting}[]
\NormalTok{herois\_info }\SpecialCharTok{\%\textgreater{}\%} 
  \FunctionTok{pull}\NormalTok{(peso) }\SpecialCharTok{\%\textgreater{}\%} 
  \FunctionTok{summary}\NormalTok{()}
\end{Highlighting}
\end{Shaded}

\begin{verbatim}
   Min. 1st Qu.  Median    Mean 3rd Qu.    Max.    NA's 
 -99,00  -99,00   62,00   43,86   90,00  900,00       2 
\end{verbatim}

  Observe que existem valores \texttt{NA} em peso.

\begin{Shaded}
\begin{Highlighting}[]
\NormalTok{herois\_info }\SpecialCharTok{\%\textgreater{}\%} 
  \FunctionTok{summarize}\NormalTok{(}
    \AttributeTok{minimo =} \FunctionTok{min}\NormalTok{(peso),}
    \AttributeTok{maximo =} \FunctionTok{max}\NormalTok{(peso)}
\NormalTok{  )}
\end{Highlighting}
\end{Shaded}

\begin{verbatim}
# A tibble: 1 x 2
  minimo maximo
   <dbl>  <dbl>
1     NA     NA
\end{verbatim}

  Para ignorar os valores \texttt{NA} nas funções \texttt{min()} e
  \texttt{max()}:

\begin{Shaded}
\begin{Highlighting}[]
\NormalTok{herois\_info }\SpecialCharTok{\%\textgreater{}\%} 
  \FunctionTok{summarize}\NormalTok{(}
    \AttributeTok{minimo =} \FunctionTok{min}\NormalTok{(peso, }\AttributeTok{na.rm =} \ConstantTok{TRUE}\NormalTok{),}
    \AttributeTok{maximo =} \FunctionTok{max}\NormalTok{(peso, }\AttributeTok{na.rm =} \ConstantTok{TRUE}\NormalTok{)}
\NormalTok{  )}
\end{Highlighting}
\end{Shaded}

\begin{verbatim}
# A tibble: 1 x 2
  minimo maximo
   <dbl>  <dbl>
1    -99    900
\end{verbatim}

  Quantos pesos negativos existem?

\begin{Shaded}
\begin{Highlighting}[]
\NormalTok{herois\_info }\SpecialCharTok{\%\textgreater{}\%} \FunctionTok{count}\NormalTok{(peso }\SpecialCharTok{\textless{}} \DecValTok{0}\NormalTok{)}
\end{Highlighting}
\end{Shaded}

\begin{verbatim}
# A tibble: 3 x 2
  `peso < 0`     n
  <lgl>      <int>
1 FALSE        495
2 TRUE         237
3 NA             2
\end{verbatim}

  Substituindo por \texttt{NA}:

\begin{Shaded}
\begin{Highlighting}[]
\NormalTok{herois\_info }\OtherTok{\textless{}{-}}\NormalTok{ herois\_info }\SpecialCharTok{\%\textgreater{}\%} 
  \FunctionTok{mutate}\NormalTok{(}
    \AttributeTok{peso =} \FunctionTok{if\_else}\NormalTok{(}
\NormalTok{      peso }\SpecialCharTok{\textless{}} \DecValTok{0}\NormalTok{,}
      \ConstantTok{NA\_real\_}\NormalTok{,}
\NormalTok{      peso}
\NormalTok{    )}
\NormalTok{  )}
\end{Highlighting}
\end{Shaded}

  Como \texttt{if\_else} (com \emph{underscore}) é exigente, precisamos
  passar \texttt{NA\_real\_} em vez de \texttt{NA}.

  Situação atual:

\begin{Shaded}
\begin{Highlighting}[]
\NormalTok{herois\_info }\SpecialCharTok{\%\textgreater{}\%} 
  \FunctionTok{pull}\NormalTok{(peso) }\SpecialCharTok{\%\textgreater{}\%} 
  \FunctionTok{summary}\NormalTok{()}
\end{Highlighting}
\end{Shaded}

\begin{verbatim}
   Min. 1st Qu.  Median    Mean 3rd Qu.    Max.    NA's 
    2,0    61,0    81,0   112,3   108,0   900,0     239 
\end{verbatim}

  \end{tcolorbox}
\item
  Qual é o peso médio de todos os heróis? Ignore os valores \texttt{NA}.

  \begin{tcolorbox}[enhanced jigsaw, coltitle=black, colbacktitle=quarto-callout-tip-color!10!white, title=\textcolor{quarto-callout-tip-color}{\faLightbulb}\hspace{0.5em}{Resposta}, toprule=.15mm, leftrule=.75mm, opacityback=0, colback=white, arc=.35mm, breakable, bottomtitle=1mm, left=2mm, toptitle=1mm, titlerule=0mm, rightrule=.15mm, bottomrule=.15mm, opacitybacktitle=0.6, colframe=quarto-callout-tip-color-frame]

\begin{Shaded}
\begin{Highlighting}[]
\NormalTok{herois\_info }\SpecialCharTok{\%\textgreater{}\%} \FunctionTok{pull}\NormalTok{(peso) }\SpecialCharTok{\%\textgreater{}\%} \FunctionTok{mean}\NormalTok{(}\AttributeTok{na.rm =} \ConstantTok{TRUE}\NormalTok{)}
\end{Highlighting}
\end{Shaded}

\begin{verbatim}
[1] 112,2525
\end{verbatim}

  \end{tcolorbox}
\item
  Qual é a altura média de todos os heróis? Ignore os valores
  \texttt{NA}.

  \begin{tcolorbox}[enhanced jigsaw, coltitle=black, colbacktitle=quarto-callout-tip-color!10!white, title=\textcolor{quarto-callout-tip-color}{\faLightbulb}\hspace{0.5em}{Resposta}, toprule=.15mm, leftrule=.75mm, opacityback=0, colback=white, arc=.35mm, breakable, bottomtitle=1mm, left=2mm, toptitle=1mm, titlerule=0mm, rightrule=.15mm, bottomrule=.15mm, opacitybacktitle=0.6, colframe=quarto-callout-tip-color-frame]

\begin{Shaded}
\begin{Highlighting}[]
\NormalTok{herois\_info }\SpecialCharTok{\%\textgreater{}\%} \FunctionTok{pull}\NormalTok{(altura) }\SpecialCharTok{\%\textgreater{}\%} \FunctionTok{mean}\NormalTok{(}\AttributeTok{na.rm =} \ConstantTok{TRUE}\NormalTok{)}
\end{Highlighting}
\end{Shaded}

\begin{verbatim}
[1] 186,7263
\end{verbatim}

  \end{tcolorbox}
\item
  Qual é a altura média dos heróis, por editora? Ignore os valores
  \texttt{NA}.

  \begin{tcolorbox}[enhanced jigsaw, coltitle=black, colbacktitle=quarto-callout-tip-color!10!white, title=\textcolor{quarto-callout-tip-color}{\faLightbulb}\hspace{0.5em}{Resposta}, toprule=.15mm, leftrule=.75mm, opacityback=0, colback=white, arc=.35mm, breakable, bottomtitle=1mm, left=2mm, toptitle=1mm, titlerule=0mm, rightrule=.15mm, bottomrule=.15mm, opacitybacktitle=0.6, colframe=quarto-callout-tip-color-frame]

\begin{Shaded}
\begin{Highlighting}[]
\NormalTok{herois\_info }\SpecialCharTok{\%\textgreater{}\%} 
  \FunctionTok{group\_by}\NormalTok{(editora) }\SpecialCharTok{\%\textgreater{}\%} 
  \FunctionTok{summarize}\NormalTok{(média }\OtherTok{=} \FunctionTok{mean}\NormalTok{(altura, }\AttributeTok{na.rm =} \ConstantTok{TRUE}\NormalTok{))}
\end{Highlighting}
\end{Shaded}

\begin{verbatim}
# A tibble: 3 x 2
  editora média
  <chr>   <dbl>
1 DC       181.
2 Marvel   191.
3 Outras   179.
\end{verbatim}

  \end{tcolorbox}
\item
  Quais são os $3$ heróis mais altos de cada sexo?

  \begin{tcolorbox}[enhanced jigsaw, coltitle=black, colbacktitle=quarto-callout-tip-color!10!white, title=\textcolor{quarto-callout-tip-color}{\faLightbulb}\hspace{0.5em}{Resposta}, toprule=.15mm, leftrule=.75mm, opacityback=0, colback=white, arc=.35mm, breakable, bottomtitle=1mm, left=2mm, toptitle=1mm, titlerule=0mm, rightrule=.15mm, bottomrule=.15mm, opacitybacktitle=0.6, colframe=quarto-callout-tip-color-frame]

\begin{Shaded}
\begin{Highlighting}[]
\NormalTok{herois\_info }\SpecialCharTok{\%\textgreater{}\%} 
  \FunctionTok{group\_by}\NormalTok{(sexo) }\SpecialCharTok{\%\textgreater{}\%} 
  \FunctionTok{slice\_max}\NormalTok{(altura, }\AttributeTok{n =} \DecValTok{3}\NormalTok{) }\SpecialCharTok{\%\textgreater{}\%} 
  \FunctionTok{select}\NormalTok{(nome, sexo, altura)}
\end{Highlighting}
\end{Shaded}

\begin{verbatim}
# A tibble: 11 x 3
# Groups:   sexo [3]
  nome          sexo         altura
  <chr>         <chr>         <dbl>
1 Living Brain  Desconhecido    198
2 Fabian Cortez Desconhecido    196
3 Box III       Desconhecido    193
4 Firelord      Desconhecido    193
5 Wolfsbane     Female          366
6 Rey           Female          297
# i 5 more rows
\end{verbatim}

  Como houve empates, foram mostrados $4$ de sexo desconhecido e $4$ do
  sexo feminino.

  Leia a documentação da função \texttt{slice\_max} para descobrir como
  mostrar exatamente $n$ de cada grupo. (Dica: ``empate'', em inglês, é
  ``\emph{tie}''.)

  \end{tcolorbox}
\item
  Quais são as $3$ cores de olhos mais comuns para cada sexo?

  \begin{tcolorbox}[enhanced jigsaw, coltitle=black, colbacktitle=quarto-callout-tip-color!10!white, title=\textcolor{quarto-callout-tip-color}{\faLightbulb}\hspace{0.5em}{Resposta}, toprule=.15mm, leftrule=.75mm, opacityback=0, colback=white, arc=.35mm, breakable, bottomtitle=1mm, left=2mm, toptitle=1mm, titlerule=0mm, rightrule=.15mm, bottomrule=.15mm, opacitybacktitle=0.6, colframe=quarto-callout-tip-color-frame]

\begin{Shaded}
\begin{Highlighting}[]
\NormalTok{herois\_info }\SpecialCharTok{\%\textgreater{}\%} 
  \FunctionTok{group\_by}\NormalTok{(sexo) }\SpecialCharTok{\%\textgreater{}\%} 
  \FunctionTok{count}\NormalTok{(olhos, }\AttributeTok{sort =} \ConstantTok{TRUE}\NormalTok{) }\SpecialCharTok{\%\textgreater{}\%} 
  \FunctionTok{slice\_head}\NormalTok{(}\AttributeTok{n =} \DecValTok{3}\NormalTok{)}
\end{Highlighting}
\end{Shaded}

\begin{verbatim}
# A tibble: 9 x 3
# Groups:   sexo [3]
  sexo         olhos     n
  <chr>        <chr> <int>
1 Desconhecido <NA>     10
2 Desconhecido blue      6
3 Desconhecido red       5
4 Female       blue     76
5 Female       green    43
6 Female       <NA>     41
# i 3 more rows
\end{verbatim}

  \end{tcolorbox}
\item
  Liste, por editora, as quantidades de heróis do bem, do mal, e
  neutros.

  \begin{tcolorbox}[enhanced jigsaw, coltitle=black, colbacktitle=quarto-callout-tip-color!10!white, title=\textcolor{quarto-callout-tip-color}{\faLightbulb}\hspace{0.5em}{Resposta}, toprule=.15mm, leftrule=.75mm, opacityback=0, colback=white, arc=.35mm, breakable, bottomtitle=1mm, left=2mm, toptitle=1mm, titlerule=0mm, rightrule=.15mm, bottomrule=.15mm, opacitybacktitle=0.6, colframe=quarto-callout-tip-color-frame]

\begin{Shaded}
\begin{Highlighting}[]
\NormalTok{herois\_info }\SpecialCharTok{\%\textgreater{}\%} 
  \FunctionTok{group\_by}\NormalTok{(editora) }\SpecialCharTok{\%\textgreater{}\%} 
  \FunctionTok{count}\NormalTok{(lado)}
\end{Highlighting}
\end{Shaded}

\begin{verbatim}
# A tibble: 11 x 3
# Groups:   editora [3]
  editora lado        n
  <chr>   <chr>   <int>
1 DC      bad        59
2 DC      good      142
3 DC      neutral    13
4 DC      <NA>        1
5 Marvel  bad       115
6 Marvel  good      259
# i 5 more rows
\end{verbatim}

  ou

\begin{Shaded}
\begin{Highlighting}[]
\NormalTok{herois\_info }\SpecialCharTok{\%\textgreater{}\%} 
  \FunctionTok{group\_by}\NormalTok{(editora, lado) }\SpecialCharTok{\%\textgreater{}\%} 
  \FunctionTok{summarize}\NormalTok{(}\FunctionTok{n}\NormalTok{())}
\end{Highlighting}
\end{Shaded}

\begin{verbatim}
`summarise()` has grouped output by 'editora'. You can override
using the `.groups` argument.
\end{verbatim}

\begin{verbatim}
# A tibble: 11 x 3
# Groups:   editora [3]
  editora lado    `n()`
  <chr>   <chr>   <int>
1 DC      bad        59
2 DC      good      142
3 DC      neutral    13
4 DC      <NA>        1
5 Marvel  bad       115
6 Marvel  good      259
# i 5 more rows
\end{verbatim}

  \end{tcolorbox}
\item
  Quantas raças diferentes existem?

  \begin{tcolorbox}[enhanced jigsaw, coltitle=black, colbacktitle=quarto-callout-tip-color!10!white, title=\textcolor{quarto-callout-tip-color}{\faLightbulb}\hspace{0.5em}{Resposta}, toprule=.15mm, leftrule=.75mm, opacityback=0, colback=white, arc=.35mm, breakable, bottomtitle=1mm, left=2mm, toptitle=1mm, titlerule=0mm, rightrule=.15mm, bottomrule=.15mm, opacitybacktitle=0.6, colframe=quarto-callout-tip-color-frame]

\begin{Shaded}
\begin{Highlighting}[]
\NormalTok{herois\_info }\SpecialCharTok{\%\textgreater{}\%} 
  \FunctionTok{pull}\NormalTok{(raça) }\SpecialCharTok{\%\textgreater{}\%} 
  \FunctionTok{n\_distinct}\NormalTok{()}
\end{Highlighting}
\end{Shaded}

\begin{verbatim}
[1] 62
\end{verbatim}

  ou (mostrando os nomes das raças e as quantidades de heróis por raça)

\begin{Shaded}
\begin{Highlighting}[]
\NormalTok{herois\_info }\SpecialCharTok{\%\textgreater{}\%} 
  \FunctionTok{count}\NormalTok{(raça)}
\end{Highlighting}
\end{Shaded}

\begin{verbatim}
# A tibble: 62 x 2
  raça          n
  <chr>     <int>
1 Alien         7
2 Alpha         5
3 Amazon        2
4 Android       9
5 Animal        4
6 Asgardian     5
# i 56 more rows
\end{verbatim}

  ou

\begin{Shaded}
\begin{Highlighting}[]
\NormalTok{herois\_info }\SpecialCharTok{\%\textgreater{}\%} 
  \FunctionTok{group\_by}\NormalTok{(raça) }\SpecialCharTok{\%\textgreater{}\%} 
  \FunctionTok{summarise}\NormalTok{(}\FunctionTok{n}\NormalTok{())}
\end{Highlighting}
\end{Shaded}

\begin{verbatim}
# A tibble: 62 x 2
  raça      `n()`
  <chr>     <int>
1 Alien         7
2 Alpha         5
3 Amazon        2
4 Android       9
5 Animal        4
6 Asgardian     5
# i 56 more rows
\end{verbatim}

  \end{tcolorbox}
\item
  Qual a quantidade de raças diferentes de cada editora?

  \begin{tcolorbox}[enhanced jigsaw, coltitle=black, colbacktitle=quarto-callout-tip-color!10!white, title=\textcolor{quarto-callout-tip-color}{\faLightbulb}\hspace{0.5em}{Resposta}, toprule=.15mm, leftrule=.75mm, opacityback=0, colback=white, arc=.35mm, breakable, bottomtitle=1mm, left=2mm, toptitle=1mm, titlerule=0mm, rightrule=.15mm, bottomrule=.15mm, opacitybacktitle=0.6, colframe=quarto-callout-tip-color-frame]

\begin{Shaded}
\begin{Highlighting}[]
\NormalTok{herois\_info }\SpecialCharTok{\%\textgreater{}\%} 
  \FunctionTok{group\_by}\NormalTok{(editora) }\SpecialCharTok{\%\textgreater{}\%} 
  \FunctionTok{summarise}\NormalTok{(}\AttributeTok{n =} \FunctionTok{n\_distinct}\NormalTok{(raça))}
\end{Highlighting}
\end{Shaded}

\begin{verbatim}
# A tibble: 3 x 2
  editora     n
  <chr>   <int>
1 DC         30
2 Marvel     32
3 Outras     23
\end{verbatim}

  \end{tcolorbox}
\item
  \textbf{DESAFIO:} Liste as raças que só pertencem a uma única editora.

  Existem várias maneiras de fazer isto. Experimente várias, até achar
  uma que seja mais elegante.

  \begin{tcolorbox}[enhanced jigsaw, coltitle=black, colbacktitle=quarto-callout-tip-color!10!white, title=\textcolor{quarto-callout-tip-color}{\faLightbulb}\hspace{0.5em}{Resposta}, toprule=.15mm, leftrule=.75mm, opacityback=0, colback=white, arc=.35mm, breakable, bottomtitle=1mm, left=2mm, toptitle=1mm, titlerule=0mm, rightrule=.15mm, bottomrule=.15mm, opacitybacktitle=0.6, colframe=quarto-callout-tip-color-frame]

  \begin{itemize}
  \item
    Maneira simples, usando contagem:

\begin{Shaded}
\begin{Highlighting}[]
\NormalTok{herois\_info }\SpecialCharTok{\%\textgreater{}\%} 
  \FunctionTok{group\_by}\NormalTok{(raça) }\SpecialCharTok{\%\textgreater{}\%} 
  \FunctionTok{summarise}\NormalTok{(}\AttributeTok{n\_ed =} \FunctionTok{n\_distinct}\NormalTok{(editora)) }\SpecialCharTok{\%\textgreater{}\%} 
  \FunctionTok{filter}\NormalTok{(n\_ed }\SpecialCharTok{==} \DecValTok{1}\NormalTok{)}
\end{Highlighting}
\end{Shaded}

\begin{verbatim}
# A tibble: 47 x 2
  raça        n_ed
  <chr>      <int>
1 Alpha          1
2 Amazon         1
3 Asgardian      1
4 Bizarro        1
5 Bolovaxian     1
6 Clone          1
# i 41 more rows
\end{verbatim}
  \item
    Maneira repetitiva, manual:

\begin{Shaded}
\begin{Highlighting}[]
\NormalTok{racas\_marvel }\OtherTok{\textless{}{-}}\NormalTok{ herois\_info }\SpecialCharTok{\%\textgreater{}\%} 
  \FunctionTok{filter}\NormalTok{(editora }\SpecialCharTok{==} \StringTok{\textquotesingle{}Marvel\textquotesingle{}}\NormalTok{) }\SpecialCharTok{\%\textgreater{}\%} 
  \FunctionTok{select}\NormalTok{(raça) }\SpecialCharTok{\%\textgreater{}\%} 
  \FunctionTok{unique}\NormalTok{()}

\NormalTok{racas\_dc }\OtherTok{\textless{}{-}}\NormalTok{ herois\_info }\SpecialCharTok{\%\textgreater{}\%} 
  \FunctionTok{filter}\NormalTok{(editora }\SpecialCharTok{==} \StringTok{\textquotesingle{}DC\textquotesingle{}}\NormalTok{) }\SpecialCharTok{\%\textgreater{}\%} 
  \FunctionTok{select}\NormalTok{(raça) }\SpecialCharTok{\%\textgreater{}\%} 
  \FunctionTok{unique}\NormalTok{()}

\NormalTok{racas\_outras }\OtherTok{\textless{}{-}}\NormalTok{ herois\_info }\SpecialCharTok{\%\textgreater{}\%} 
  \FunctionTok{filter}\NormalTok{(editora }\SpecialCharTok{==} \StringTok{\textquotesingle{}Outras\textquotesingle{}}\NormalTok{) }\SpecialCharTok{\%\textgreater{}\%} 
  \FunctionTok{select}\NormalTok{(raça) }\SpecialCharTok{\%\textgreater{}\%} 
  \FunctionTok{unique}\NormalTok{()}
\end{Highlighting}
\end{Shaded}

    Exclusivas da Marvel:

\begin{Shaded}
\begin{Highlighting}[]
\NormalTok{racas\_marvel }\SpecialCharTok{\%\textgreater{}\%} 
  \FunctionTok{setdiff}\NormalTok{(racas\_dc) }\SpecialCharTok{\%\textgreater{}\%} 
  \FunctionTok{setdiff}\NormalTok{(racas\_outras) }\SpecialCharTok{\%\textgreater{}\%} 
  \FunctionTok{arrange}\NormalTok{(raça)}
\end{Highlighting}
\end{Shaded}

\begin{verbatim}
# A tibble: 17 x 1
  raça          
  <chr>         
1 Asgardian     
2 Clone         
3 Cosmic Entity 
4 Eternal       
5 Flora Colossus
6 Frost Giant   
# i 11 more rows
\end{verbatim}

    Exclusivas da DC:

\begin{Shaded}
\begin{Highlighting}[]
\NormalTok{racas\_dc }\SpecialCharTok{\%\textgreater{}\%} 
  \FunctionTok{setdiff}\NormalTok{(racas\_marvel) }\SpecialCharTok{\%\textgreater{}\%} 
  \FunctionTok{setdiff}\NormalTok{(racas\_outras) }\SpecialCharTok{\%\textgreater{}\%} 
  \FunctionTok{arrange}\NormalTok{(raça)}
\end{Highlighting}
\end{Shaded}

\begin{verbatim}
# A tibble: 17 x 1
  raça           
  <chr>          
1 Amazon         
2 Bizarro        
3 Bolovaxian     
4 Czarnian       
5 Gorilla        
6 Human-Vuldarian
# i 11 more rows
\end{verbatim}

    Exclusivas de outras editoras:

\begin{Shaded}
\begin{Highlighting}[]
\NormalTok{racas\_outras }\SpecialCharTok{\%\textgreater{}\%} 
  \FunctionTok{setdiff}\NormalTok{(racas\_dc) }\SpecialCharTok{\%\textgreater{}\%} 
  \FunctionTok{setdiff}\NormalTok{(racas\_marvel) }\SpecialCharTok{\%\textgreater{}\%} 
  \FunctionTok{arrange}\NormalTok{(raça)}
\end{Highlighting}
\end{Shaded}

\begin{verbatim}
# A tibble: 13 x 1
  raça              
  <chr>             
1 Alpha             
2 Dathomirian Zabrak
3 Gungan            
4 Human / Clone     
5 Human-Vulcan      
6 Icthyo Sapien     
# i 7 more rows
\end{verbatim}
  \item
    Mesma maneira, mas usando uma função:

\begin{Shaded}
\begin{Highlighting}[]
\NormalTok{racas\_exclusivas }\OtherTok{\textless{}{-}} \ControlFlowTok{function}\NormalTok{(x) \{}

\NormalTok{  esta\_editora }\OtherTok{\textless{}{-}}\NormalTok{ herois\_info }\SpecialCharTok{\%\textgreater{}\%} 
    \FunctionTok{filter}\NormalTok{(editora }\SpecialCharTok{==}\NormalTok{ x) }\SpecialCharTok{\%\textgreater{}\%} 
    \FunctionTok{select}\NormalTok{(raça) }\SpecialCharTok{\%\textgreater{}\%} 
    \FunctionTok{unique}\NormalTok{()}

\NormalTok{  outras\_editoras }\OtherTok{\textless{}{-}}\NormalTok{ herois\_info }\SpecialCharTok{\%\textgreater{}\%} 
    \FunctionTok{filter}\NormalTok{(editora }\SpecialCharTok{!=}\NormalTok{ x) }\SpecialCharTok{\%\textgreater{}\%} 
    \FunctionTok{select}\NormalTok{(raça) }\SpecialCharTok{\%\textgreater{}\%} 
    \FunctionTok{unique}\NormalTok{()}

\NormalTok{  esta\_editora }\SpecialCharTok{\%\textgreater{}\%}
    \FunctionTok{setdiff}\NormalTok{(outras\_editoras) }\SpecialCharTok{\%\textgreater{}\%} 
    \FunctionTok{arrange}\NormalTok{(raça)}
\NormalTok{\}}
\end{Highlighting}
\end{Shaded}

\begin{Shaded}
\begin{Highlighting}[]
\FunctionTok{racas\_exclusivas}\NormalTok{(}\StringTok{\textquotesingle{}Marvel\textquotesingle{}}\NormalTok{)}
\end{Highlighting}
\end{Shaded}

\begin{verbatim}
# A tibble: 17 x 1
  raça          
  <chr>         
1 Asgardian     
2 Clone         
3 Cosmic Entity 
4 Eternal       
5 Flora Colossus
6 Frost Giant   
# i 11 more rows
\end{verbatim}

\begin{Shaded}
\begin{Highlighting}[]
\FunctionTok{racas\_exclusivas}\NormalTok{(}\StringTok{\textquotesingle{}DC\textquotesingle{}}\NormalTok{)}
\end{Highlighting}
\end{Shaded}

\begin{verbatim}
# A tibble: 17 x 1
  raça           
  <chr>          
1 Amazon         
2 Bizarro        
3 Bolovaxian     
4 Czarnian       
5 Gorilla        
6 Human-Vuldarian
# i 11 more rows
\end{verbatim}

\begin{Shaded}
\begin{Highlighting}[]
\FunctionTok{racas\_exclusivas}\NormalTok{(}\StringTok{\textquotesingle{}Outras\textquotesingle{}}\NormalTok{)}
\end{Highlighting}
\end{Shaded}

\begin{verbatim}
# A tibble: 13 x 1
  raça              
  <chr>             
1 Alpha             
2 Dathomirian Zabrak
3 Gungan            
4 Human / Clone     
5 Human-Vulcan      
6 Icthyo Sapien     
# i 7 more rows
\end{verbatim}
  \item
    Maneira complicada, usando \texttt{join}:

\begin{Shaded}
\begin{Highlighting}[]
\NormalTok{herois\_info }\SpecialCharTok{\%\textgreater{}\%} 
  \FunctionTok{select}\NormalTok{(raça, editora) }\SpecialCharTok{\%\textgreater{}\%} 
  \FunctionTok{group\_by}\NormalTok{(raça) }\SpecialCharTok{\%\textgreater{}\%} 
  \FunctionTok{summarise}\NormalTok{(}\AttributeTok{n\_editoras =} \FunctionTok{n\_distinct}\NormalTok{(editora)) }\SpecialCharTok{\%\textgreater{}\%} 
  \FunctionTok{filter}\NormalTok{(n\_editoras }\SpecialCharTok{==} \DecValTok{1}\NormalTok{) }\SpecialCharTok{\%\textgreater{}\%} 
  \FunctionTok{inner\_join}\NormalTok{(herois\_info, }\AttributeTok{by =} \StringTok{\textquotesingle{}raça\textquotesingle{}}\NormalTok{) }\SpecialCharTok{\%\textgreater{}\%} 
  \FunctionTok{select}\NormalTok{(raça, editora) }\SpecialCharTok{\%\textgreater{}\%} 
  \FunctionTok{unique}\NormalTok{() }\SpecialCharTok{\%\textgreater{}\%} 
  \FunctionTok{arrange}\NormalTok{(editora)}
\end{Highlighting}
\end{Shaded}

\begin{verbatim}
# A tibble: 47 x 2
  raça            editora
  <chr>           <chr>  
1 Amazon          DC     
2 Bizarro         DC     
3 Bolovaxian      DC     
4 Czarnian        DC     
5 Gorilla         DC     
6 Human-Vuldarian DC     
# i 41 more rows
\end{verbatim}
  \end{itemize}

  \end{tcolorbox}
\end{enumerate}

\section{\texorpdfstring{Examinando \emph{tibbles}
intermediárias}{Examinando tibbles intermediárias}}\label{examinando-tibbles-intermediuxe1rias}

\begin{itemize}
\item
  \href{https://github.com/daranzolin/ViewPipeSteps}{O pacote
  \texttt{ViewPipeSteps}} serve para exibir (no console ou em
  \emph{tabs} no RStudio) as \emph{tibbles} que são resultados de cada
  passo em uma sequência de comandos montada com o \emph{pipe}
  \texttt{\%\textgreater{}\%}.
\item
  Instale o pacote com o comando

\begin{Shaded}
\begin{Highlighting}[]
\FunctionTok{install.packages}\NormalTok{(}\StringTok{"ViewPipeSteps"}\NormalTok{)}
\end{Highlighting}
\end{Shaded}
\item
  Carregue o pacote com

\begin{Shaded}
\begin{Highlighting}[]
\FunctionTok{library}\NormalTok{(ViewPipeSteps)}
\end{Highlighting}
\end{Shaded}
\item
  Para exibir, no console, as \emph{tibbles} intermediárias, acrescente
  \texttt{print\_pipe\_steps(all\ =\ TRUE)} após o último passo do
  \emph{pipe}:

  ::: \{.cell layout-align=``center''\}

\begin{Shaded}
\begin{Highlighting}[]
\NormalTok{resultado }\OtherTok{\textless{}{-}}\NormalTok{ bb\_tidy }\SpecialCharTok{\%\textgreater{}\%} 
  \FunctionTok{group\_by}\NormalTok{(artista, musica) }\SpecialCharTok{\%\textgreater{}\%} 
  \FunctionTok{summarize}\NormalTok{(}\AttributeTok{media =} \FunctionTok{sum}\NormalTok{(pos)}\SpecialCharTok{/}\FunctionTok{max}\NormalTok{(semana), }\AttributeTok{.groups =} \StringTok{\textquotesingle{}drop\textquotesingle{}}\NormalTok{) }\SpecialCharTok{\%\textgreater{}\%} 
  \FunctionTok{arrange}\NormalTok{(media) }\SpecialCharTok{\%\textgreater{}\%} 
  \FunctionTok{print\_pipe\_steps}\NormalTok{(}\AttributeTok{all =} \ConstantTok{TRUE}\NormalTok{)}
\end{Highlighting}
\end{Shaded}

  ::: \{.cell-output .cell-output-stderr\}

\begin{verbatim}
1. bb_tidy
\end{verbatim}

  :::

  ::: \{.cell-output .cell-output-stdout\}

\begin{verbatim}
# A tibble: 5.307 x 5
  artista musica                  entrou     semana   pos
  <chr>   <chr>                   <date>      <int> <dbl>
1 2 Pac   Baby Don't Cry (Keep... 2000-02-26      1    87
2 2 Pac   Baby Don't Cry (Keep... 2000-02-26      2    82
3 2 Pac   Baby Don't Cry (Keep... 2000-02-26      3    72
4 2 Pac   Baby Don't Cry (Keep... 2000-02-26      4    77
5 2 Pac   Baby Don't Cry (Keep... 2000-02-26      5    87
6 2 Pac   Baby Don't Cry (Keep... 2000-02-26      6    94
# i 5.301 more rows
\end{verbatim}

  :::

  ::: \{.cell-output .cell-output-stderr\}

\begin{verbatim}
2. group_by(artista, musica)
\end{verbatim}

  :::

  ::: \{.cell-output .cell-output-stdout\}

\begin{verbatim}
# A tibble: 5.307 x 5
  artista musica                  entrou     semana   pos
  <chr>   <chr>                   <date>      <int> <dbl>
1 2 Pac   Baby Don't Cry (Keep... 2000-02-26      1    87
2 2 Pac   Baby Don't Cry (Keep... 2000-02-26      2    82
3 2 Pac   Baby Don't Cry (Keep... 2000-02-26      3    72
4 2 Pac   Baby Don't Cry (Keep... 2000-02-26      4    77
5 2 Pac   Baby Don't Cry (Keep... 2000-02-26      5    87
6 2 Pac   Baby Don't Cry (Keep... 2000-02-26      6    94
# i 5.301 more rows
\end{verbatim}

  :::

  ::: \{.cell-output .cell-output-stderr\}

\begin{verbatim}
3. summarize(media = sum(pos)/max(semana), .groups = "drop")
\end{verbatim}

  :::

  ::: \{.cell-output .cell-output-stdout\}

\begin{verbatim}
# A tibble: 317 x 3
  artista      musica                  media
  <chr>        <chr>                   <dbl>
1 2 Pac        Baby Don't Cry (Keep...  85.4
2 2Ge+her      The Hardest Part Of ...  90  
3 3 Doors Down Kryptonite               26.5
4 3 Doors Down Loser                    67.1
5 504 Boyz     Wobble Wobble            56.2
6 98^0         Give Me Just One Nig...  37.6
# i 311 more rows
\end{verbatim}

  :::

  ::: \{.cell-output .cell-output-stderr\}

\begin{verbatim}
4. arrange(media)
\end{verbatim}

  :::

  ::: \{.cell-output .cell-output-stdout\}

\begin{verbatim}
# A tibble: 317 x 3
  artista                          musica                  media
  <chr>                            <chr>                   <dbl>
1 "Santana"                        Maria, Maria             10.5
2 "Madonna"                        Music                    13.5
3 "N'Sync"                         Bye Bye Bye              14.3
4 "Elliott, Missy \"Misdemeanor\"" Hot Boyz                 14.3
5 "Destiny's Child"                Independent Women Pa...  14.8
6 "Iglesias, Enrique"              Be With You              15.8
# i 311 more rows
\end{verbatim}

  ::: :::
\item
  Para exibir as \emph{tibbles} intermediárias em \emph{tabs} do RStudio
  (como com a função \texttt{View()}), você pode usar o \emph{addin}
  \texttt{viewPipeChain}, que também faz parte deste pacote. Veja o
  exemplo \href{https://github.com/daranzolin/ViewPipeSteps}{no site do
  pacote}.
\end{itemize}



\end{document}
